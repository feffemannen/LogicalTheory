% Part:sets-functions-relations
% Chapter: size-of-sets
% Section: introduction

\documentclass[../../../include/open-logic-section]{subfiles}

\begin{document}

\olfileid{sfr}{siz}{int}

\olsection{Introduction}

When Georg Cantor developed set theory in the 1870s, one of his aims
was to make palatable the idea of an infinite collection---an actual
infinity, as the medievals would say.  A key part of this was his
treatment of the \emph{size} of different sets. If $a$, $b$ and $c$ are
all distinct, then the set $\{a, b, c\}$ is intuitively \emph{larger}
than $\{a, b\}$. But what about infinite sets? Are they all as large
as each other? It turns out that they are not.

The first important idea here is that of an enumeration.  We can
list every finite set by listing all its !!{element}s.  For some
infinite sets, we can also list all their !!{element}s if we allow the
list itself to be infinite. Such sets are called !!{enumerable}.
Cantor's surprising result, which we will fully understand by the end
of this chapter, was that some infinite sets are not !!{enumerable}.

\end{document}
