% Part:sets-functions-relations
% Chapter: size-of-sets
% Section: pairing-alt

\documentclass[../../../include/open-logic-section]{subfiles}

\begin{document}

\olfileid{sfr}{siz}{pai-alt}

\olsection{An Alternative Pairing Function}

\begin{explain}
There are other enumerations of $\Nat^2$ that make it easier to
figure out what their inverses are. Here is one. Instead of
visualizing the enumeration in an array, start with the list of
positive integers associated with (initially) empty spaces. Imagine
filling these spaces successively with pairs $\tuple{n,m}$ as follows.
Starting with the pairs that have~$0$ in  the first place (i.e., pairs
$\tuple{0,m}$), put the first (i.e., $\tuple{0,0}$) in the first empty
place, then skip an empty space, put the second (i.e., $\tuple{0,2}$)
in the next empty place, skip one again, and so forth. The
(incomplete) beginning of our enumeration now looks like this
\[\small
\begin{array}{@{}c c c c c c c c c c c@{}}
\mathbf 1 & \mathbf 2 & \mathbf 3 & \mathbf 4 & \mathbf 5 & \mathbf 6 & \mathbf 7 & \mathbf 8 & \mathbf 9 & \mathbf{10} & \dots \\ \\
\tuple{0,1} &  & \tuple{0,2} &  & \tuple{0,3} & & \tuple{0,4} &  & \tuple{0,5} &  & \dots \\
\end{array}
\]
Repeat this with pairs $\tuple{1,m}$ for the place that still remain
empty, again skipping every other empty place:
\[\small
\begin{array}{@{}c c c c c c c c c c c@{}}
\mathbf 1 & \mathbf 2 & \mathbf 3 & \mathbf 4 & \mathbf 5 & \mathbf 6 & \mathbf 7 & \mathbf 8 & \mathbf 9 & \mathbf{10} & \dots \\ \\
\tuple{0,0} & \tuple{1,0} & \tuple{0,1} &  & \tuple{0,2} & \tuple{1,1} & 
\tuple{0,3} & & \tuple{0,4} &  \tuple{1,2} & \dots \\
\end{array}
\]
Enter pairs $\tuple{2,m}$, $\tuple{2,m}$, etc., in the same way. Our
completed enumeration thus starts like this:
\[\small
\begin{array}{@{}cc c c c c c c c c c@{}}
\mathbf 1 & \mathbf 2 & \mathbf 3 & \mathbf 4 & \mathbf 5 & \mathbf 6 & \mathbf 7 & \mathbf 8 & \mathbf 9 & \mathbf{10} & \dots \\ \\
\tuple{0,0} & \tuple{1,0} & \tuple{0,1} & \tuple{2,0}  & \tuple{0,2} & 
\tuple{1,1} & \tuple{0,3} & \tuple{3,0}  & \tuple{0,4} &  \tuple{1,2} & \dots \\
\end{array}
\]
If we number the cells in the array above according to this
enumeration, we will not find a neat zig-zag line, but this
arrangement:
\[
\begin{array}{ c | c | c | c | c | c | c | c }
& \mathbf 0 & \mathbf 1 & \mathbf 2 & \mathbf 3 & \mathbf 4 & \mathbf 5 & \dots \\
\hline
\mathbf 0 & 1 & 3 & 5 & 7 & 9 & 11 & \dots \\
\hline
\mathbf 1 & 2 & 6 & 10 & 14 & 18 & \dots & \dots \\
\hline
\mathbf 2 & 4 & 12 & 20 & 28 & \dots & \dots & \dots \\
\hline
\mathbf 3 & 8 & 24 & 40 & \dots & \dots & \dots & \dots \\
\hline
\mathbf 4 & 16 & 48 & \dots & \dots & \dots & \dots & \dots \\
\hline
\mathbf 5 & 32 & \dots & \dots & \dots & \dots & \dots & \dots \\
\hline
\vdots & \vdots & \vdots & \vdots & \vdots & \vdots & \vdots & \ddots\\
\end{array}
\]
We can see that the pairs in row~$0$ are in the odd numbered places of
our enumeration, i.e., pair $\tuple{0,m}$ is in place $2m+1$; pairs in
the second row, $\tuple{1,m}$, are in places whose number is the
double of an odd number, specifically,  $2 \cdot (2m+1)$; pairs in the
third row, $\tuple{2,m}$, are in places whose number is four times an
odd number, $4 \cdot (2m+1)$; and so on. The factors of $(2m+1)$ for
each row, $1$, $2$, $4$, $8$, \dots, are exactly the powers of~$2$:
$1= 2^0$, $2 = 2^1$, $4 = 2^2$, $8 = 2^3$, \dots\@ In fact, the
relevant exponent is always the first member of the pair in
question. Thus, for pair $\tuple{n,m}$ the factor is $2^n$.  This
gives us the general formula: $2^n \cdot (2m+1)$. However, this is a
mapping of pairs to \emph{positive} integers, i.e., $\tuple{0,0}$ has
position~$1$. If we want to begin at position~$0$ we must subtract~$1$
from the result. This gives us:
\end{explain}

\begin{ex}
The function $h\colon \Nat^2 \to \Nat$ given by
\[
h(n,m) = 2^n (2m+1) - 1
\]
is a pairing function for the set of pairs of natural numbers~$\Nat^2$.
\end{ex}

\begin{explain}
Accordingly, in our second enumeration of $\Nat^2$, the pair
$\tuple{0,0}$ has code $h(0,0) = 2^0(2\cdot 0+1) - 1 = 0$;
$\tuple{1,2}$ has code $2^{1} \cdot (2 \cdot 2 + 1) - 1 = 2
\cdot 5 - 1 = 9$; $\tuple{2,6}$ has code $2^{2} \cdot (2
\cdot 6 + 1) - 1 = 51$.
\end{explain}

Sometimes it is enough to encode pairs of natural numbers~$\Nat^2$
without requiring that the encoding is surjective. Such encodings have
inverses that are only partial functions. 

\begin{ex}
The function $j\colon \Nat^2 \to \Nat^+$ given by
\[
j(n,m) = 2^n3^m
\]
is !!a{injective} function $\Nat^2 \to \Nat$.
\end{ex}

\end{document}