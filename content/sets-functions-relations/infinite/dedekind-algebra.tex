% Part: sets-functions-relations
% Chapter: infinite
% Section: dedekind
%
\documentclass[../../../include/open-logic-section]{subfiles}

\begin{document}

\olfileid{sfr}{infinite}{dedekind}
\olsection{Dedekind Algebras}

We not only want natural numbers to be infinite; we want them to have
certain (algebraic) properties: they need to behave well under
addition, multiplication, and so forth. 

Dedekind's idea was to take the idea of the \emph{successor function}
as basic, and then characterise the numbers as those with the
following properties:
\begin{enumerate}
	\item There is a number, $0$, which is not the successor of any number
	\\i.e., $0 \notin \ran{s}$
	\\i.e., $\forall x\ s(x) \neq 0$
	\item Distinct numbers have distinct successors 
	\\i.e., $s$ is !!a{injection}
	\\i.e., $\forall x \forall y (s(x) = s(y) \lif x = y)$
	\item\ollabel{repeatedapplication} Every number is obtained from
	$0$ by repeated applications of the successor function.
\end{enumerate}
The first two conditions are easy to deal with using first-order logic
(see above). But we cannot deal with \olref{repeatedapplication} just
using first-order logic. Dedekind's breakthrough was to reformulate
condition \olref{repeatedapplication}, set-theoretically, as follows:
\begin{enumerate}
	\item[3$'$.] The natural numbers are the smallest set that is
	\emph{closed under the successor function}: that is, if we apply
	$s$ to any !!{element} of the set, we obtain another !!{element}
	of the set.
\end{enumerate}
But we shall need to spell this out slowly.

\begin{defn}\ollabel{Closure}
	For any function $f$, the set $X$ is $f$-\emph{closed} {iff}
	$(\forall x \in X)f(x) \in X$. Now define, for any $o$:
	$$\closureofunder{f}{o} = \bigcap\Setabs{X}{o \in X\text{ and }X
\text{ is $f$-closed}}$$
\end{defn}

So $\closureofunder{f}{o}$ is the intersection of all the $f$-closed
sets with $o$ as !!a{element}. Intuitively, then,
$\closureofunder{f}{o}$ is the \emph{smallest} $f$-closed set with $o$
as !!a{element}. This next result makes that intuitive thought
precise;
\begin{lem}\ollabel{closureproperties}
	For any function $f$ and any $o \in A$:
	\begin{enumerate}
		\item\ollabel{closurehaselem} $o \in \closureofunder{f}{o}$; and
		\item\ollabel{closureclosed} $\closureofunder{f}{o}$ is $f$-closed; and
		\item\ollabel{closuresmallest} if $X$ is $f$-closed and $o \in
		X$, then $\closureofunder{f}{o} \subseteq X$
	\end{enumerate}
\end{lem}

\begin{proof}
Note that there is at least one $f$-closed set, namely $\ran{f}\cup
\{o\}$. So $\closureofunder{f}{o}$, the intersection of \emph{all}
such sets, exists. We must now check
\olref{closurehaselem}--\olref{closuresmallest}.

\olref{closurehaselem}. $o \in \closureofunder{f}{o}$ as it is an
intersection of sets which all have $o$ as !!a{element}. 

\olref{closureclosed}. Let $X$ be $f$-closed with $o \in X$. If $x \in
\closureofunder{f}{o}$, then $x \in X$, and now $f(x) \in X$ as $X$ is
$f$-closed, so $f(x) \in \closureofunder{f}{o}$.

\olref{closuresmallest}. This follows from the general fact that if $X
\in C$ then $\bigcap C \subseteq X$.
\end{proof}

Using this, we can say:

\begin{defn}
A \emph{Dedekind algebra} is a set $A$ together with a function $f
\colon A \to A$ and some $o \in A$  such that:
	\begin{enumerate}
		\item \ollabel{ded:proper} $o \notin \ran{f}$
		\item \ollabel{ded:injection} $f$ is !!a{injection}
		\item \ollabel{ded:closure} $A = \closureofunder{f}{o}$
	\end{enumerate}
\end{defn}

Since $A = \closureofunder{f}{o}$, our earlier result tells us that
$A$ is the smallest $f$-closed set with $o$ as !!a{element}. Clearly a
Dedekind algebra is Dedekind infinite; just look at clauses
\olref{ded:proper} and \olref{ded:injection} of the definition. But
the more exciting fact is that any Dedekind infinite set can be turned
into a Dedekind algebra. 

\begin{thm}\ollabel{thm:DedekindInfiniteAlgebra}
If there is a Dedekind infinite set, then there is a Dedekind algebra.
\end{thm}

\begin{proof}
Let $D$ be Dedekind infinite. So there is an injection $g \colon D \to
D$ and an element $o  \in D \setminus \ran{g}$. Now let $A =
\closureofunder{g}{o}$, and note that $o \in A$. Let $f =
\funrestrictionto{g}{A}$. We will show that this constitutes a Dedekind
algebra. 

Concerning \olref{ded:proper}: $o \notin \ran{g}$ and $\ran{f}
\subseteq \ran{g}$ so $o\notin \ran{f}$.

Concerning \olref{ded:injection}: $g$ is an injection on $D$; so $f
\subseteq g$ must be an injection.

Concerning \olref{ded:closure}: Let $o \in B$. By
\olref{closureproperties}, if $B \subsetneq A$, then $B$ is not
$g$-closed. So $B$ is not $f$-closed either, as $f =
\funrestrictionto{g}{A}$. So $A$ is the \emph{smallest} $f$-closed set
with $o$ as an element, i.e., $A = \closureofunder{f}{o}$.
\end{proof}

\end{document}