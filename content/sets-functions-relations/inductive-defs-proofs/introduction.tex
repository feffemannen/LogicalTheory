% Part: sets-functions-relations
% Chapter: induction
% Section: induction

\documentclass[../../include/open-logic-section]{subfiles}

\begin{document}

\olfileid{sfr}{ind}{int}

\olsection{Introduction}

\begin{explain}
Induction is a commonly-used proof technique of mathematics in logic.
You may be familiar with it from math, in which you do induction on
natural numbers. As it turns out, there are lots of different kinds of
mathematical and logical sets of objects that lend themselves to induction.

In general, what induction allows one to do is prove a universal claim; that
is, show that every object of a certain kind has some property. In particular,
induction is often useful where a ``universal introduction'' method of proof
is too complicated. Induction only works on mathematical objects that are
constructed in special ways: if every element in the set is either basic or is
built up from basic elements, then we can use induction on it. More precisely:
\end{explain}

\begin{defn}
A set $S$ is \emph{closed} under a function $f$ iff whenever $x \in
S$, then $f(x) \in S$.
\end{defn}

\begin{explain}
For example, the natural numbers ($\mathbb{N}$) are closed under the
successor function $f(x) = x+1$. If $x$ is a natural number, so is
$x+1$. The natural numbers are not, however, closed under the
predecessor function $f(x) = x-1$, since $0$ is a natural number, but
$-1$ is not.
\end{explain}

\begin{defn}
A property $P$ is said to be \emph{preserved under} a function $f(x)$
when, for any object $a$ in the domain of $f$, $P(a)$ implies
$P(f(a))$ (if $a$ has property $P$, then so does $f(a)$).
\end{defn}

\begin{explain}
The property of ``being an even number'' is preserved under the
function $f(x) = x+2$, since if $x$ is even, so is $x+2$. The property
of being an even number is not preserved under the successor function,
since if $x$ is even, $x+1$ is odd.

Induction works on natural numbers because every natural number can be
reached from 0 by applying the successor function a finite number of
times. This is characteristic of any set upon which we can do
induction.
\end{explain}

\begin{defn}[Induction on $\Nat$]
If 0 has some property $P$, and if $P$ is preserved under the
successor function, then every natural number has property $P$.
\end{defn}

\begin{ex} The sum of the first $n$ natural numbers is $\frac{n(n+1)}{2}$. \\

\textbf{Base case.} The sum of the first 0 natural numbers is $0=
\frac{0\cdot 1}{2}$.\\

\textbf{Inductive step.} Suppose the property holds for $k$. We will
show that it holds for $k+1$. In other words, suppose that $P(k)$
holds, that is,
\[ 0 + 1 + \ldots + k = \frac{k(k+1)}{2} \]
We will show that $P(k+1)$ holds, i.e. that
\[ 0 + 1 + \ldots + k + (k+1) = \frac{(k+1)(k+2)}{2} \]
by using the assumption that $P(k)$.

\begin{align*}
0 + 1 + \ldots + k &= \frac{k(k+1)}{2} \tag{Assumption} \\
0 + 1 + \ldots + k + (k+1) &= \frac{k(k+1)}{2} + (k+1)
\tag{Add $k+1$ to both sides} \\
&= \frac{k(k+1) + 2(k+1)}{2} \\
&= \frac{2k + k + 2k +2}{2} \\
&=\frac{(k+1)(k+2)}{2}
\end{align*}
Hence, the inductive step goes through, and we have proved the claim.
\end{ex}

\begin{explain}
For !!{formula}s, the basic elements are the atomic !!{formula}s. More
complex !!{formula}s are obtained by joining together less-complicated
!!{formula}s using operators. Each operator can be seen as
corresponding to a function of !!{formula}s that returns a new
!!{formula} joining these two together with the operator in
question. For example, the function that joins two !!{formula}s $!!^a$
and $!B$ into a conjunction can be written as $\mathcal E _\land (!!^a,
!B) = !!^a \land !B$. We can call these \emph{formula-building
  functions}. The set of !!{formula}s is closed these functions:
whenever $!!^a$ and $!B$ are !!{formula}s, so are $\lnot !!^a$, $!!^a \land
!B$, etc.
\end{explain}

\begin{defn}[Induction on !!{formula}s]
If every atomic !!{formula} has property $P$ and $P$ is preserved
under the !!{formula}-building functions, then every !!{formula} has
property $P$.
\end{defn}

\begin{explain}
This definition suggests a ``recipe'' for inductive proofs on
!!{formula}s. If we are asked to show that every !!{formula} has
property $P$, follow the following steps:\\

\textbf{Base Case.} Let $!!^a$ be an atomic !!{formula}. [\ldots]
Therefore, $!!^a$ has property $P$.

\textbf{Inductive Step.} Let $!!^a$ and $!B$ be !!{formula}s, both of
which have the property $P$.

Case $\lnot$: [\ldots] Therefore, $\lnot !!^a$ has property $P$.

Case $\land$: [\ldots] Therefore, $!!^a \land !B$ has property $P$.

Case $\lor$: [\ldots] Therefore, $!!^a \lor !B$ has property $P$.

Case $\lif$: [\ldots] Therefore, $!!^a \lif !B$ has property $P$.

Case $\liff$: [\ldots] Therefore, $!!^a \liff !B$ has property $P$.

Case $\lforall[]$: [\ldots] Therefore, $\lforall[x] !!^a$ has property $P$.

Case $\lexists[]$: [\ldots] Therefore, $\lexists[x] !!^a$ has property $P$. \\

Therefore, every !!{formula} has property $P$.
\end{explain}

\end{document}
