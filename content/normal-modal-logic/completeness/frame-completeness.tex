% Part: normal-modal-logic
% Chapter: completeness
% Section: frame-completeness

\documentclass[../../../include/open-logic-section]{subfiles}

\begin{document}

\olfileid{nml}{com}{fra}

\olsection{Frame Completeness}

The completeness theorem for \Log{K} can be extended to other modal
systems, once we show that the canonical model for a given logic has
the corresponding frame property.

\begin{thm}\ollabel{thm:completeframeprops}
  If a normal modal logic $\Sigma$ contains one of the !!{formula}s on the
  left-hand side of \olref{tab:correspondencetable},
  then the canonical model for~$\Sigma$ has the corresponding property
  on the right-hand side.
\end{thm}

\begin{table}[htp]
  \centering
    \begin{tabular}{| l || l |}
      \hline
      \emph{If $\Sigma$ contains \dots} & \emph{\dots the canonical
        model for $\Sigma$ is:} \\
      \hline \hline
      \Ax{D}:\;\; $\Box!A \lif \Diamond !A$ & \quad serial; \\
      \hline
      \Ax{T}:\;\; $\Box!A \lif !A$ &\quad  reflexive;\\
      \hline
      \Ax{B}: \;\;$!A \lif \Box\Diamond!A$ &\quad  symmetric; \\
      \hline
      \Ax{4}: \;\; $\Box!A \lif \Box\Box!A$ & \quad transitive; \\
      \hline 
      \Ax{5}: \;\; $\Diamond !A \lif \Box\Diamond!A$& \quad euclidean.\\
      \hline
    \end{tabular}
    \caption{Basic correspondence facts.}
    \ollabel{tab:correspondencetable}
\end{table}

\begin{proof}
  We take each of these up in turn.

  Suppose $\Sigma$ contains \Ax{D}, and let $\Delta \in W^\Sigma$; we
  need to show that there is a $\Delta'$ such that $R^\Sigma
  \Delta\Delta'$. It suffices to show that $\Box^{-1}\Delta$ is
  $\Sigma$-consistent, for then by Lindenbaum's Lemma, there is a
  complete $\Sigma$-consistent set $\Delta' \supseteq
  \Box^{-1}\Delta$, and by definition of $R^\Sigma$
  \iftag{prvBox}{}{and \olref[mod]{lem:box-iff-diamond} }we have
  $R^\Sigma \Delta\Delta'$. So, suppose for contradiction that
  $\Box^{-1}\Delta$ is \emph{not} $\Sigma$-consistent, i.e.,
  $\Box^{-1}\Delta \Proves[\Sigma] \lfalse$. By \olref[mod]{lem:box2},
  $\Delta \Proves[\Sigma] \Box\lfalse$, and since $\Sigma$ contains
  \Ax{D}, also $\Delta \Proves[\Sigma] \Diamond\lfalse$. But $\Sigma$
  is normal, so $\Sigma \Proves \lnot\Diamond \lfalse$
  (\olref[prf][nor]{prop:notDiamondBot}), whence also $\Delta
  \Proves[\Sigma] \lnot\Diamond \lfalse$, against the consistency
  of~$\Delta$.

  Now suppose $\Sigma$ contains \Ax{T}, and let $\Delta \in
  W^\Sigma$. We want to show $R^\Sigma \Delta\Delta$,
  i.e.,\iftag{prvBox}{}{ by \olref[mod]{lem:box-iff-diamond},}
  $\Box^{-1}\Delta \subseteq \Delta$. But if $\Box!A \in \Delta$ then
  by \Ax{T} also $!A \in \Delta$, as desired.

  Now suppose $\Sigma$ contains \Ax{B}, and suppose $R^\Sigma
  \Delta\Delta'$ for $\Delta$, $\Delta' \in W^\Sigma$. We need to show
  that $R^\Sigma \Delta'\Delta$, i.e.,\iftag{prvBox}{
    $\Box^{-1}\Delta' \subseteq \Delta$. By
    \olref[mod]{lem:box-iff-diamond}, this is equivalent to}{}
  $\Diamond\Delta \subseteq \Delta'$. So suppose $!A \in \Delta$. By
  \Ax{B}, also $\Box\Diamond!A \in \Delta$. By the hypothesis that
  $R^\Sigma \Delta\Delta'$\iftag{prvBox}{}{ and
    \olref[mod]{lem:box-iff-diamond}}, we have that $\Box^{-1}\Delta
  \subseteq \Delta'$, and hence $\Diamond!A \in \Delta'$, as required.

  Now suppose $\Sigma$ contains \Ax{4}, and suppose $R^\Sigma
  \Delta_1\Delta_2$ and $R^\Sigma \Delta_2\Delta_3$. We need to show
  $R^\Sigma \Delta_1\Delta_3$. From the hypothesis\iftag{prvBox}{}{ and
    \olref[mod]{lem:box-iff-diamond}} we have both $\Box^{-1}\Delta_1
  \subseteq \Delta_2$ and $\Box^{-1}\Delta_2 \subseteq \Delta_3$. In
  order to show $R^\Sigma \Delta_1\Delta_3$ it suffices to show
  $\Box^{-1}\Delta_1 \subseteq \Delta_3$. So let $!B \in
  \Box^{-1}\Delta_1$, i.e., $\Box!B \in \Delta_1$. By \Ax{4}, also
  $\Box\Box!B \in \Delta_1$ and by hypothesis we get, first, that
  $\Box!B \in \Delta_2$ and, second, that $!B \in \Delta_3$, as
  desired.

  Now suppose $\Sigma$ contains \Ax{5}, suppose $R^\Sigma
  \Delta_1\Delta_2$ and $R^\Sigma \Delta_1\Delta_3$. We need to show
  $R^\Sigma \Delta_2\Delta_3$. The first hypothesis gives
  $\Box^{-1}\Delta_1 \subseteq \Delta_2$\iftag{prvBox}{}{ by
    \olref[mod]{lem:box-iff-diamond}}, and the second hypothesis is
  equivalent to $\Diamond\Delta_3 \subseteq \Delta_2$\iftag{prvBox}{,
    by \olref[mod]{lem:box-iff-diamond}}{}.  To show $R^\Sigma
  \Delta_2\Delta_3$\iftag{prvBox}{, by
    \olref[mod]{lem:box-iff-diamond}, it suffices to}{ we have to}
  show $\Diamond\Delta_3 \subseteq \Delta_2$. So let $\Diamond!A \in
  \Diamond\Delta_3$, i.e., $!A \in \Delta_3$. By the second hypothesis
  $\Diamond!A \in \Delta_1$ and by \Ax{5}, $\Box\Diamond!A \in
  \Delta_1$ as well. But now the first hypothesis gives $\Diamond!A \in
  \Delta_2$, as desired.
  \end{proof}

As a corollary we obtain completeness results for a number of
systems. For instance, we know that $\Log{S5} = \Log{KT5} =
\Log{KTB4}$ is complete with respect to the class of all reflexive
euclidean models, which is the same as the class of all reflexive,
symmetric and transitive models.

\begin{thm}\ollabel{thm:generaldet}
  Let $\mClass{C}_\Ax{D}$, $\mClass{C}_\Ax{T}$,
  $\mClass{C}_\Ax{B}$, $\mClass{C}_\Ax{4}$, and
  $\mClass{C}_\Ax{5}$ be the class of all serial, reflexive,
  symmetric, transitive, and euclidean models (respectively). Then for
  any schemas $!A_1$, \dots, $!A_n$ among \Ax{D},
  \Ax{T}, \Ax{B}, \Ax{4}, and \Ax{5}, the system
  $\Log{K}!A_1 \dots !A_n$ is determined by the
  class of models $\mClass{C} = \mClass{C}_{!A_1} \cap \dots
  \cap \mClass{C}_{!A_n}$. 
\end{thm}

\begin{prop}
  Let $\Sigma$ be a normal modal logic; then:
  \begin{enumerate}
  \item\ollabel{prop:anotherfive-a}%
    If $\Sigma$ contains the schema $\Diamond!A \lif \Box 
    !A$ then the canonical model for $\Sigma$ is partially functional. 
  \item If $\Sigma$ contains the schema $\Diamond!A \liff \Box 
    !A$ then the canonical model for $\Sigma$ is functional. 
  \item If $\Sigma$ contains the schema $\Box\Box!A \lif \Box 
    !A$ then the canonical model for $\Sigma$ is weakly dense. 
  \end{enumerate}
(see \olref[frd][acc]{tab:anotherfive} for definitions of these frame
  properties).
\end{prop}

\begin{proof}
  \begin{enumerate}
  \item Suppose that $\Sigma$ contains the schema $\Diamond !A \lif
    \Box !A$, to show that $R^\Sigma$ is partially functional we need
    to prove that for any $\Delta_1$, $\Delta_2$, $\Delta_3 \in
    W^\Sigma$, if $R^\Sigma \Delta_1\Delta_2$ and $R^\Sigma
    \Delta_1\Delta_3$ then $\Delta_2=\Delta_3$. Since $R^\Sigma
    \Delta_1\Delta_2$ we have $\Box^{-1}\Delta_1 \subseteq \Delta_2$
    and since $R^\Sigma \Delta_1\Delta_3$ also $\Box^{-1}\Delta_1
    \subseteq \Delta_3$\iftag{prvBox}{}{, both by
      \olref[mod]{lem:box-iff-diamond}}. The identity
    $\Delta_2=\Delta_3$ will follow if we can establish the two
    inclusions $\Delta_2 \subseteq \Delta_3$ and $\Delta_3 \subseteq
    \Delta_2$. For the first inclusion, let $!A \in \Delta_2$; then
    $\Diamond!A \in \Delta_1$, and by the schema and deductive closure
    of $\Delta_1$ also $\Box!A \in \Delta_1$, whence by the hypothesis
    that $R^\Sigma \Delta_1\Delta_3$, $!A \in \Delta_3$. The second
    inclusion is similar.

  \item This follows immediately from part~\olref{prop:anotherfive-a}
    and the seriality proof in \olref{thm:completeframeprops}.

  \item Suppose $\Sigma$ contains the schema $\Box\Box!A \lif \Box!A$
    and to show that $R^\Sigma$ is weakly dense, let $R^\Sigma
    \Delta_1\Delta_2$. We need to show that there is a complete
    $\Sigma$-consistent set $\Delta_3$ such that $R^\Sigma
    \Delta_1\Delta_3$ and $R^\Sigma \Delta_3\Delta_2$. Let:
    \[
    \Gamma = \Box^{-1}\Delta_1 \cup \Diamond\Delta_2.
    \]
    It suffices to show that $\Gamma$ is $\Sigma$-consistent, for then
    by Lindenbaum's Lemma it can be extended to a complete
    $\Sigma$-consistent set~$\Delta_3$ such that $\Box^{-1}\Delta_1
    \subseteq \Delta_3$ and $\Diamond\Delta_2 \subseteq \Delta_3$,
    i.e., $R^\Sigma \Delta_1\Delta_3$\iftag{prvBox}{}{ (by
      \olref[mod]{lem:box-iff-diamond})} and $R^\Sigma
    \Delta_3\Delta_2$\iftag{prvBox}{ (by
      \olref[mod]{lem:box-iff-diamond})}{}.

    Suppose for contradiction that $\Gamma$ is not consistent. Then
    there are !!{formula}s $\Box!A_1$, \dots, $\Box!A_n \in \Delta_1$
    and $!B_1$, \dots,~$!B_m \in \Delta_2$ such that \[!A_1, \dots,
    !A_n, \Diamond!B_1, \dots, \Diamond!B_m \Proves[\Sigma]
    \lfalse.\] Since $\Diamond (!B_1 \land \dots \land !B_m) \to
    (\Diamond!B_1 \land \dots \land \Diamond!B_m)$ is !!{derivable} in
    every normal modal logic, we argue as follows, contradicting the
    consistency of~$\Delta_2$:
    \begin{align*}
      !A_1, \dots, !A_n, & \Diamond!B_1, \dots,
      \Diamond!B_m \Proves[\Sigma] \lfalse \\
      !A_1, \dots,!A_n & \Proves[\Sigma] (\Diamond!B_1 \land
      \dots \land \Diamond!B_m) \lif \lfalse\\
      & \qquad\text{by the deduction theorem}\\
      & \qquad\text{\olref[prf][prp]{prop:derivabilityfacts}\olref[prf][prp]{prop:derivabilityfacts-deduction}, and \Taut} \\
      !A_1, \dots,!A_n & \Proves[\Sigma]
      \Diamond(!B_1 \land
      \dots \land !B_m) \lif \lfalse\\
      & \qquad \text{since $\Sigma$ is normal} \\
      !A_1, \dots,!A_n & \Proves[\Sigma]
      \lnot\Diamond (!B_1 \land \dots \land !B_m)\\
      & \qquad\text{by \PL} \\
      !A_1, \dots,!A_n & \Proves[\Sigma]
      \Box\lnot (!B_1 \land \dots \land !B_m)\\
      & \qquad\text{$\Box\lnot$ for $\lnot\Diamond$} \\
      \Box!A_1, \dots,\Box!A_n
      & \Proves[\Sigma] \Box\Box \lnot (!B_1 \land \dots \land !B_m)\\
      & \qquad\text{by \olref[mod]{lem:box1}} \\
      \Box!A_1, \dots,\Box!A_n
      & \Proves[\Sigma]
      \Box\lnot (!B_1 \land \dots \land !B_m)\\
      &\qquad\text{by schema $\Box\Box!A \lif \Box!A$} \\
      \Delta_1 & \Proves[\Sigma]
      \Box\lnot (!B_1 \land \dots \land !B_m)\\
      &\qquad\text{by monotonicity, \olref[prf][prp]{prop:derivabilityfacts}\olref[prf][prp]{prop:derivabilityfacts-monotonicity}} \\
      & \Box\lnot (!B_1 \land \dots \land !B_m) \in \Delta_1\\
      &\qquad\text{by deductive closure}; \\
      & \lnot (!B_1 \land \dots \land !B_m) \in \Delta_2\\
      &\qquad \text{since }
      R^\Sigma \Delta_1\Delta_2. 
    \end{align*}
  \end{enumerate}
\end{proof}

On the strength of these examples, one might think that every
system~$\Sigma$ of modal logic is \emph{complete}, in the sense that
it proves every formula which is valid in every frame in which every
theorem of $\Sigma$ is valid. Unfortunately, there are many systems
that are not complete in this sense.

\end{document}
