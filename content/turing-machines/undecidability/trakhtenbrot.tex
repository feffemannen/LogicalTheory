% Part: turing-machines
% Chapter: undecidability
% Section: trakhtenbrot

\documentclass[../../../include/open-logic-section]{subfiles}

\begin{document}

\olfileid{tur}{und}{tra}
\olsection{Trakthenbrot's Theorem}

\begin{explain}
  In \olref[rep]{sec} we defined !!{sentence}s $!T(M,w)$ and~$!E(M,w)$
  for a Turing machine~$M$ and input string~$w$. Then we showed in
  \olref[ver]{lem:valid-if-halt} and \olref[ver]{lem:halt-if-valid}
  that $!T(M,w) \lif !E(M,w)$ is valid iff $T$~started on input~$w$
  eventually halts. Since the Halting Problem is undecidable, this
  implies that validity and satisfiability of !!{sentence}s of
  first-order logic is undecidable
  (\cref{tur:und:uns:thm:decision-prob,tur:und:uns:cor:undecidable-sat}).

  But validity and satisfiability of sentences is defined for
  arbitrary !!{structure}s, finite or infinite. You might suspect that
  it is easier to decide if !!a{sentence} is satisfiable in a finite
  !!{structure} (or valid in all finite !!{structure}s). We can
  improve the proof of the unsolvability of the decision so that it
  shows this is not the case.

  First, if you go back to the proof of
  \olref[ver]{lem:halt-if-valid}, you'll see that what we did there is
  produce a model~$\Struct M$ of~$!T(M,w)$ which describes exactly
  what machine~$M$ does when started on input~$w$.  The domain of that
  model was~$\Nat$, i.e., infinite. But if $M$ actually halts on
  input~$w$, we can build a finite model~$\Struct M'$ in the same way.
  Suppose $M$ started on input~$w$ halts after~$k$ steps. Take as
  domain $\Domain{M'}$ the set $\{0, \dots, n\}$, where $n$ is the
  larger of~$k$ and the length of~$w$, and let
  \[
    \Assign{\prime}{M'}(x) = 
    \begin{cases}
      x + 1 &\text{if $x < n$}\\
      n &\text{otherwise.}
    \end{cases}
  \]
  Otherwise $\Struct{M'}$ is defined just like~$\Struct{M}$. By the
  definition of~$\Struct{M'}$, just like in the proof of
  \olref[ver]{lem:halt-if-valid}, $\Sat{M'}{!T(M,w)}$.  And since we
  assumed that $M$ halts on input~$w$, $\Sat{M'}{!E(M,w)}$. So,
  $\Struct{M'}$ is a finite model of~$!T(M,w) \land !E(M,w)$ (note
  that we've replaced $\lif$ by~$\land$). We are halfway to a proof:
  if $M$ halts on input~$w$, then $!T(M,e) \land !E(M,w)$ has a finite
  model. Unfortunately, the ``only if'' direction does not hold.  For
  instance, if $M$ after $n$~steps is in state~$q$ and reads a
  symbol~$\sigma$, and $\delta(q,\sigma) = \tuple{q,\sigma,\TMstay}$,
  then the configuration after $n+1$ steps is exactly the same as the
  configuration after $n$~steps (same state, same head position, same
  tape contents). But the machine never halts; it's in an infinite
  loop.  The corresponding !!{structure}~$\Struct{M'}$ above satisfies
  $!T(M,w)$ but not~$!E(M,w)$. (In it, the values of $\num{n+l}$ are
  all the same, so it is finite). But by changing~$!T(M,w)$ suitable
  we can rule out !!{structure}s like this.
\end{explain}

Consider the !!{sentence}s describing the operation of the Turing
machine~$M$ on input $w = \sigma_{i_1}\dots\sigma_{i_k}$:
\begin{enumerate}
\item Axioms describing numbers and~$<$ (just like in the definition
of~$!T(M,w)$ in \olref[rep]{sec}).
\item Axioms describing the input configuration: just like in the definition
of~$!T(M,w)$.
\item Axioms describing the transition from one configuration to
  the next:

For the following, let $!A(x, y)$ be as before, and let
\[
  !B(y) \ident \lforall[x][(x < y \lif \eq/[x][y])].
\]
\begin{enumerate}
\item \ollabel{rep-right} For every instruction $\delta(q_i, \sigma) =
  \tuple{q_j, \sigma', \TMright}$, the !!{sentence}:
\begin{align*}
& \lforall[x][\lforall[y][(
   (\Obj Q_{q_i}(x, y) \land \Obj S_{\sigma}(x, y)) \lif {}]] \\
&\qquad   (\Obj Q_{q_j}(x', y') \land \Obj S_{\sigma'}(x, y') \land
!A(x, y) \land !B(y')))
\end{align*}
In other words, the same as the corresponding !!{sentence}
in~$!T(M,w)$, except we add $!B(y')$ at the end. ($!B(y')$~ensures
that the number $y'$ of the ``next'' configuration is different from
all previous numbers $\Obj 0$, $\Obj 0'$, \dots)

\item \ollabel{rep-left} For every instruction $\delta(q_i, \sigma) =
  \tuple{q_j, \sigma', \TMleft}$, the !!{sentence}
\begin{align*}
& \lforall[x][\lforall[y][
    ((\Obj Q_{q_i}(x', y) \land \Obj S_{\sigma}(x', y)) \lif {}]]\\
& \qquad   (\Obj Q_{q_j}(x, y') \land \Obj S_{\sigma'}(x', y') \land
!A(x, y))) \land {}\\
& \lforall[y][((\Obj Q_{q_i}(\Obj 0, y) \land \Obj S_{\sigma}(\Obj 0,
    y)) \lif {}]\\
& \qquad (\Obj Q_{q_j}(\Obj 0, y') \land \Obj S_{\sigma'}(\Obj 0,
  y') \land !A(\Obj 0, y) \land !B(y')))
\end{align*}
\item \ollabel{rep-stay} For every instruction $\delta(q_i, \sigma) =
  \tuple{q_j, \sigma', \TMstay}$, the !!{sentence}:
\begin{align*}
& \lforall[x][\lforall[y][(
   (\Obj Q_{q_i}(x, y) \land \Obj S_{\sigma}(x, y)) \lif {}]] \\
&\qquad (\Obj Q_{q_j}(x, y') \land \Obj S_{\sigma'}(x, y') \land
!A(x, y) \land !B(y')))
\end{align*}
\end{enumerate}
% \item Sentences that express consistency conditions:
% \begin{enumerate}
%   \item\ollabel{rep-con-symb}%
%   !!^a{sentence} that says that no square can have more than one
%   symbol on it at any given time:
%   \[\lforall[x][\lforall[y][(\Obj S_{\sigma}(x, y) \lif \lnot \Obj
%   S_{\sigma'}(x, y))]],\]
%   for every pair $\sigma \neq \sigma'$ of tape symbols of~$T$.
%   \item\ollabel{rep-con-state}%
%   !!^a{sentence} that says that $M$ cannot be in more than one state
%   at any given time:
%   \[\lforall[x][\lforall[y][(\Obj Q_{q}(x, y) \lif
%   \lforall[z][\lnot \Obj
%   Q_{q'}(z, y))]],\]
%   for every pair $q \neq q'$ of states of~$T$.
%   \item\ollabel{rep-con-tape}%
%   !!^a{sentence} that says that $M$ cannot be on more than one tape
%   square at any given time:
%   \[\lforall[x][\lforall[y][(\Obj Q_{q}(x, y) \lif
%   \lforall[z][\eq/[z][y]\lif \lnot \Obj
%   Q_{q'}(z, y))]],\]
%   for every pair $q$, $q'$ of states of~$T$.
% \end{enumerate}
\end{enumerate}
Let $!T'(M, w)$ be the conjunction of all the above !!{sentence}s for Turing
machine~$M$ and input~$w$.

\begin{lem}\ollabel{lem:halts-sat}
  If $M$ started on input~$w$ halts, then $!T'(M,w) \land !E(M,w)$ has
  a finite model.
\end{lem}

\begin{proof}
  Let $\Struct{M'}$ be as in the proof of
  \olref[ver]{lem:halt-if-valid}, except
  \begin{align*}
    \Domain{M'} & = \{0, \dots, n\}\\
    \Assign{\prime}{M'}(x) & = 
    \begin{cases}
      x + 1 &\text{if $x < n$}\\
      n &\text{otherwise,}
    \end{cases}
  \end{align*}
  where $n = \max(k,\len{w})$ and $k$~is the least number such that
  $M$ started on input~$w$ has halted after~$k$ steps. We leave the
  verification that $\Sat{M'}{!T(M,w) \land E(M,w)}$ as an exercise.
\end{proof}

\begin{prob}
  Complete the proof of \olref[tur][und][tra]{lem:halts-sat} by
  proving that $\Sat{M'}{!T(M,w) \land E(M,w)}$.
\end{prob}

\begin{lem}\ollabel{lem:sat-halts}
  If $!T'(M,w) \land !E(M,w)$ has a finite model, then $M$ started on
  input~$w$ halts.
\end{lem}

\begin{proof}
  We show the contrapositive. Suppose that $M$ started on~$w$ does not
  halt. If $!T'(M,w) \land !E(M,w)$ has no model at all, we are done.
  So assume $\Struct{M}$ is a model of~$!T(M,w) \land !E(M, w)$. We
  have to show that it cannot be finite.
  
  We can prove, just like in \olref[ver]{lem:config}, that if~$M$,
  started on input~$w$, has not halted after~$n$ steps, then $!T'(M,w)
  \Entails !C(M, w, n) \land !B(\num{n})$. Since $M$ started on
  input~$w$ does not halt, $!T'(M,w) \Entails !C(M, w, n) \land
  !B(\num{n})$ for all~$n \in \Nat$. Note that by
  \olref[rep]{prop:mlessk}, $!T'(M,w) \Entails \num{k} < \num{n}$ for
  all~$k < n$. Also $!B(\num{n}) \Entails \num{k} < \num{n} \lif
  \eq/[\num{k}][\num{n}]$. So, $\Sat{M}{\eq/[\num{k}][\num{n}]}$ for
  all~$k < n$, i.e., the infinitely many terms~$\num{k}$ must all have
  different values in~$\Struct{M}$. But this requires that
  $\Domain{M}$ be infinite, so $\Struct{M}$ cannot be a finite model
  of~$!T(M,w) \land !E(M, w)$.
\end{proof}

\begin{prob}
  Complete the proof of \olref[tur][und][tra]{lem:sat-halts} by
  proving that if~$M$, started on input~$w$, has not halted after~$n$
  steps, then $!T'(M,w) \Entails !B(\num{n})$.
\end{prob}

\begin{thm}[Trakthenbrot's Theorem]
  \ollabel{thm:trakhtenbrodt}
  It is undecidable if an arbitrary !!{sentence} of first-order logic has
  a finite model (i.e., is finitely satisfiable).
\end{thm}

\begin{proof}
  Suppose there were a Turing machine~$F$ that decides the finite
  satisfiability problem. Then given any Turing machine $M$ and
  input~$w$, we could compute the sentence~$!T'(M,w) \land !E(M,w)$,
  and use~$F$ to decide if it has a finite model. By
  \cref{tur:und:tra:lem:halts-sat,tur:und:tra:lem:sat-halts}, it does iff $M$
  started on input~$w$ halts. So we could use $F$ to solve the halting
  problem, which we know is unsolvable.
\end{proof}

\begin{cor}\ollabel{cor:fproof-incomp}%
  There can be no !!{derivation} system that is sound and complete for
  \emph{finite} validity, i.e., a !!{derivation} system which has $\Proves !B$
  iff $\Sat{M}{!B}$ for every finite !!{structure}~$\Struct{M}$.
\end{cor}

\begin{proof}
  Exercise.
\end{proof}

\begin{prob}
  Prove \olref[tur][und][tra]{cor:fproof-incomp}. Observe that $!B$ is
  satisfied in every finite !!{structure} iff $\lnot !B$ is not
  finitely satisfiable.  Explain why finite satisfiability is
  semi-decidable in the sense of
  \olref[tur][und][uns]{thm:valid-ce}. Use this to argue that if
  there were a !!{derivation} system for finite validity, then finite
  satisfiability would be decidable.
\end{prob}


\end{document}
