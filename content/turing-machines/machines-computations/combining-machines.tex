% Part: turing-machines
% Chapter: machines-computations
% Section: combining-machines

\documentclass[../../../include/open-logic-section]{subfiles}

\begin{document}

\olfileid{tur}{mac}{cmb}
\olsection{Combining Turing Machines}

\begin{explain}
The examples of Turing machines we have seen so far have been fairly
simple in nature. But in fact, any problem that can be solved with any
modern programming language can also be solved with Turing machines.
To build more complex Turing machines, it is important to convince
ourselves that we can combine them, so we can build machines to solve
more complex problems by breaking the procedure into simpler parts.
If we can find a natural way to break a complex problem down into
constituent parts, we can tackle the problem in several stages,
creating several simple Turing machines and combining them into one
machine that can solve the problem. This point is especially important
when tackling the Halting Problem in the next section.

How do we combine Turing machines $M = \tuple{Q, \Sigma, q_0, \delta}$
and~$M' = \tuple{Q', \Sigma', q_0', \delta'}$?  We now use the
configuration of the tape after $M$~has halted as the input
configuration of a run of machine~$M'$.  To get a single Turing
machine $M \frown M'$ that does this, do the following:
\begin{enumerate}
    \item Renumber (or relabel) all the states~$Q'$ of~$M'$ so that $M$
    and~$M'$ have no states in common ($Q \cap Q' = \emptyset$).
    \item The states of $M \frown M'$ are $Q \cup Q'$.
    \item The tape alphabet is $\Sigma \cup \Sigma'$.
    \item The start state is~$q_0$.
    \item The transition function is the function $\delta''$ given by:
    \[\delta''(q,\sigma) =
    \begin{cases}
      \delta(q,\sigma) & \text{if $q \in Q$}\\
      \delta'(q,\sigma) & \text{if $q \in Q'$}\\
      \tuple{q_0', \sigma, \TMstay} & \text{if $q \in Q$ and
      $\delta(q,\sigma)$ undefined}
    \end{cases}\]
\end{enumerate}
The resulting machine uses the instructions of~$M$ when it is in a
state $q \in Q$, the instructions of~$M'$ when it is in a state~$q \in
Q'$. When it is in a state $q \in Q$ and is scanning a symbol~$\sigma$
for which $M$ has no transition (i.e., $M$ would have halted), it
enters the start state of~$M'$ (and leaves the tape contents and head
position as it is).

Note that unless the machine~$M$ is disciplined, we don't know where
the tape head is when $M$~halts, so the halting configuration of~$M$
need not have the head scanning square~$1$. When combining machines,
it's important to keep this in mind.
\end{explain}

\begin{ex}
\emph{Combining Machines:} We'll design a machine which, when started
on input consisting of two blocks of~$\TMstroke$'s of length $n$
and~$m$, halts with a single block of $2(m+n)$ $\TMstroke$'s on the
tape. In order to build this machine, we can combine two machines we
are already familiar with: the addition machine, and the doubler. We
begin by drawing a state diagram for the addition machine.
\[
\begin{tikzpicture}[->,>=stealth',shorten >=1pt,auto,node distance=2.8cm,
                    semithick]
  \tikzstyle{every state}=[fill=none,draw=black,text=black]

  \node[initial,state] (A)              {$q_0$};
  \node[state]         (B) [right of=A] {$q_1$};
  \node[state]         (C) [right of=B] {$q_2$};

  \path (A) edge node {\TMtrans{\TMblank}{\TMstroke}{\TMstay}} (B)
            edge [loop above] node {\TMtrans{\TMstroke}{\TMstroke}{\TMright}} (B)
        (B) edge [loop above] node {\TMtrans{\TMstroke}{\TMstroke}{\TMright}} (B)
            edge node {\TMtrans{\TMblank}{\TMblank}{\TMleft}} (C)
        (C) edge [loop above] node {\TMtrans{\TMstroke}{\TMblank}{\TMstay}} (C);
\end{tikzpicture}
\]
Instead of halting in state~$q_2$, we want to continue operation in
order to double the output. Recall that the doubler machine erases the
first stroke in the input and writes two strokes in a separate output.
Let's add an instruction to make sure the tape head is reading the
first stroke of the output of the addition machine.
\[
\begin{tikzpicture}[->,>=stealth',shorten >=1pt,auto,node distance=2.8cm,
                    semithick]
  \tikzstyle{every state}=[fill=none,draw=black,text=black]

  \node[initial,state] (A)              {$q_0$};
  \node[state]         (B) [right of=A] {$q_1$};
  \node[state]         (C) [right of=B] {$q_2$};
  \node[state]         (D) [below left of=C] {$q_3$};
  \node[state]         (E) [below left of=D] {$q_4$};

  \path (A) edge node {\TMtrans{\TMblank}{\TMstroke}{\TMstay}} (B)
            edge [loop above] node {\TMtrans{\TMstroke}{\TMstroke}{\TMright}} (B)
        (B) edge [loop above] node {\TMtrans{\TMstroke}{\TMstroke}{\TMright}} (B)
            edge node {\TMtrans{\TMblank}{\TMblank}{\TMleft}} (C)
        (C) edge node {\TMtrans{\TMstroke}{\TMblank}{\TMleft}} (D)
        (D) edge [loop left] node {\TMtrans{\TMstroke}{\TMstroke}{\TMleft}} (D)
            edge node[left, xshift=-2mm] {\TMtrans{\TMendtape}{\TMendtape}{\TMright}} (E);
\end{tikzpicture}
\]
It is now easy to double the input---all we have to do is connect the
doubler machine onto state~$q_4$. This requires renaming the states of
the doubler machine so that they start at~$q_4$ instead
of~$q_0$---this way we don't end up with two starting states. The
final diagram should look as in \olref{fig:combined}.
\begin{figure}
\[
\begin{tikzpicture}[->,>=stealth',shorten >=1pt,auto,node distance=2.8cm,
                    semithick]
  \tikzstyle{every state}=[fill=none,draw=black,text=black]
  \node[initial,state] (A)              {$q_0$};
  \node[state]         (B) [right of=A] {$q_1$};
  \node[state]         (C) [right of=B] {$q_2$};
  \node[state]         (D) [below left of=C] {$q_3$};
  \node[state]         (E) [below left of=D] {$q_4$};
  \node[state]         (2) [right of=E] {$q_5$};
  \node[state]         (3) [right of=2] {$q_6$};
  \node[state]         (4) [below of=3] {$q_7$};
  \node[state]         (5) [left of=4]  {$q_8$};
  \node[state]         (6) [left of=5]  {$q_9$};

  \path (A) edge node {\TMtrans{\TMblank}{\TMstroke}{\TMstay}} (B)
            edge [loop above] node {\TMtrans{\TMstroke}{\TMstroke}{\TMright}} (B)
        (B) edge [loop above] node {\TMtrans{\TMstroke}{\TMstroke}{\TMright}} (B)
            edge node {\TMtrans{\TMblank}{\TMblank}{\TMleft}} (C)
        (C) edge node {\TMtrans{\TMstroke}{\TMblank}{\TMleft}} (D)
        (D) edge [loop left] node {\TMtrans{\TMstroke}{\TMstroke}{\TMleft}} (D)
            edge node[left, xshift=-2mm] {\TMtrans{\TMendtape}{\TMendtape}{\TMright}} (E)
    (E) edge node {\TMtrans{\TMstroke}{\TMblank}{\TMright}} (2)
    (2) edge [loop above] node {\TMtrans{\TMstroke}{\TMstroke}{\TMright}} (2)
      edge node {\TMtrans{\TMblank}{\TMblank}{\TMright}} (3)
    (3) edge [loop above] node {\TMtrans{\TMstroke}{\TMstroke}{\TMright}} (3)
        edge node {\TMtrans{\TMblank}{\TMstroke}{\TMright}} (4)
    (4) edge [loop below] node {\TMtrans{\TMblank}{\TMstroke}{\TMleft}} (4)
        edge node {\TMtrans{\TMstroke}{\TMstroke}{\TMleft}} (5)
    (5) edge [loop below]  node {\TMtrans{\TMstroke}{\TMstroke}{\TMleft}} (5)
        edge              node {\TMtrans{\TMblank}{\TMblank}{\TMleft}} (6)
    (6) edge [loop below] node {\TMtrans{\TMstroke}{\TMstroke}{\TMleft}} (6)
        edge              node {\TMtrans{\TMblank}{\TMblank}{\TMright}} (E);
\end{tikzpicture}
\]
\caption{Combining adder and doubler machines}
\ollabel{fig:combined}
\end{figure}
\end{ex}

\begin{prop}
    If $M$ and $M'$ are disciplined and compute the functions $f\colon
    \Nat^k \to \Nat$ and $f'\colon \Nat \to \Nat$, respectively, then
    $M \frown M'$ is disciplined and computes~$\comp{f}{f'}$.
\end{prop}

\begin{proof}
    Since $M$ is disciplined, when it halts with
    output~$f(n_1,\dots,n_k) = m$, the head is scanning square~$1$. If
    we now enter the start state of~$M'$, the machine will halt with
    output $f(m)$, again scanning square~$1$. The other conditions of
    \olref[dis]{defn:disciplined} are also satisfied.
\end{proof}

\begin{prob}
    Give a disciplined Turing machine computing $f(x) = x+2$ by taking
    the machine~$M$ from \cref{tur:mac:dis:prob:disc-succ} and
    construct $M \frown M$.
\end{prob}
\end{document}
