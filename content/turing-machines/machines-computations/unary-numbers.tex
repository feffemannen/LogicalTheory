% Part: turing-machines
% Chapter: machines-computations
% Section: unary-numbers

\documentclass[../../../include/open-logic-section]{subfiles}

\begin{document}

\olfileid{tur}{mac}{una}
\olsection{Unary Representation of Numbers}

\begin{explain}
Turing machines work on sequences of symbols written on their tape.
Depending on the alphabet a Turing machine uses, these sequences of
symbols can represent various inputs and outputs.  Of particular
interest, of course, are Turing machines which compute
\emph{arithmetical} functions, i.e., functions of natural numbers.  A
simple way to represent positive integers is by coding them as
sequences of a single symbol~$\TMstroke$.  If $n \in \Nat$, let
$\TMstroke^n$ be the empty sequence if $n = 0$, and otherwise the
sequence consisting of exactly $n$ $\TMstroke$'s.
\end{explain}

\begin{defn}[Computation]
A Turing machine~$M$ \emph{computes} the function $f\colon \Nat^n \to \Nat$ iff
$M$ halts on input
\[
\TMstroke^{k_1} \TMblank \TMstroke^{k_2} \TMblank \dots \TMblank \TMstroke^{k_n}
\]
with output $\TMstroke^{f(k_1, \dots, k_n)}$. 
\end{defn}

\begin{ex} \emph{Addition:}
Build a machine that, when given an input of two non-empty strings of
$\TMstroke$'s of length~$n$ and $m$, computes the function $f(n,m) = n
+ m$.

We want to come up with a machine that starts with two blocks of
strokes on the tape and halts with one block of strokes. We first need
a method to carry out.  The input strokes are separated by a blank, so
one method would be to write a stroke on the square containing the
blank, and erase the first (or last) stroke. This would result in a
block of~$n + m$ $\TMstroke$'s. Alternatively, we could proceed in a
similar way to the doubler machine, by erasing a stroke from the first
block, and adding one to the second block of strokes until the first
block has been removed completely. We will proceed with the former
example.
\[
\begin{tikzpicture}[->,>=stealth',shorten >=1pt,auto,node distance=2.8cm,
                    semithick]
  \tikzstyle{every state}=[fill=none,draw=black,text=black]

  \node[initial,state]         (A)              {$q_0$};
  \node[state]         (B) [right of=A] {$q_1$};
  \node[state]         (C) [right of=B] {$q_2$};

  \path (A) edge node {\TMtrans{\TMblank}{\TMstroke}{\TMright}} (B)
           edge [loop above] node {\TMtrans{\TMstroke}{\TMstroke}{\TMright}} (B)
        (B) edge [loop above] node {\TMtrans{\TMstroke}{\TMstroke}{\TMright}} (B)
            edge node {\TMtrans{\TMblank}{\TMblank}{\TMleft}} (C)
        (C) edge [loop above] node {\TMtrans{\TMstroke}{\TMblank}{\TMstay}} (C);
\end{tikzpicture}
\]
\end{ex}

\begin{prob}
Trace through the configurations of the machine for input $\tuple{3,5}$.
\end{prob}

\begin{prob}
\emph{Subtraction:} Design a Turing machine that when given an input
of two non-empty strings of strokes of length $n$ and~$m$, where $n >
m$, computes the function $f(n,m) = n - m$.
\end{prob}

\begin{prob}
\emph{Equality:} Design a Turing machine to compute the following function:
\[
\fn{equality}(x,y) = 
\begin{cases}
  \text{1} & \text{if~$x = y$} \\
  \text{0} & \text{if~$x \neq y$}
\end{cases}
\]
where~$x$ and~$y$ are integers greater than~$0$.
\end{prob}

\begin{prob}
Design a Turing machine to compute the function $\min(x,y)$ where $x$
and $y$ are positive integers represented on the tape by strings of
$\TMstroke$'s separated by a $\TMblank$. You may use additional
symbols in the alphabet of the machine.

The function $\min$ selects the smallest value from its arguments, so
$\min(3,5)=3$, $\min(20,16)=16$, and $\min(4,4)=4$, and so on.
\end{prob}

\end{document}
