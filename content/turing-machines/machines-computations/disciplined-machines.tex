% Part: turing-machines
% Chapter: machines-computations
% Section: well-behaved-machines

\documentclass[../../../include/open-logic-section]{subfiles}

\begin{document}

\olfileid{tur}{mac}{dis}
\olsection{Disciplined Machines}

\begin{explain}
In section \olref[hal]{sec}, we considered Turing machines that have a
single, designated halting state~$h$---such machines are guaranteed to
halt, if they halt at all, in state~$h$.  In this way, machines with a
single halting state are more ``disciplined'' than we allow Turing
machines in general to be.  There are other restrictions we might
impose on the behavior of Turing machines.  For instance, we also have
not prohibited Turing machines from ever erasing the tape-end marker
on square~$0$, or to attempt to move left from square~$0$. (Our
definition states that the head simply stays on square~$0$ in this
case; other definitions have the machine halt.) It is likewise
sometimes desirable to be able to assume that a Turing machine, if
it halts at all, halts on square~$1$.
\end{explain}

\begin{defn}\ollabel{defn:disciplined}
A Turing machine~$M$ is \emph{disciplined} iff
\begin{enumerate}
    \item it has a designated single halting state~$h$,
    \item it halts, if it halts at all, while scanning square~$1$,
    \item it never erases the $\TMendtape$ symbol on square~$0$, and
    \item it never attempts to move left from square~$0$.
\end{enumerate}
\end{defn}

\begin{explain}
We have already discussed that any Turing machine can be changed into
one with the same behavior but with a designated halting state. This is
done simply by adding a new state~$h$, and adding an instruction
$\delta(q, \sigma) = \tuple{h, \sigma, N}$ for any pair
$\tuple{q,\sigma}$ where the original $\delta$~is undefined.  It is
true, although tedious to prove, that any Turing machine~$M$ can be
turned into a disciplined Turing machine~$M'$ which halts on the same
inputs and produces the same output. For instance, if the Turing
machine halts and is not on square~$1$, we can add some instructions
to make the head move left until it finds the tape-end marker, then
move one square to the right, then halt.  We'll leave you to think
about how the other conditions can be dealt with.
\end{explain}

\begin{ex}
    In \olref{fig:adder-disc}, we turn the addition machine
    from \olref[una]{ex:adder} into a disciplined machine.
    \begin{figure}\[
\begin{tikzpicture}[->,>=stealth',shorten >=1pt,auto,node distance=2.8cm,
                    semithick]
  \tikzstyle{every state}=[fill=none,draw=black,text=black]

  \node[initial,state]         (A)              {$q_0$};
  \node[state]         (B) [right of=A] {$q_1$};
  \node[state]         (C) [below right of=B] {$q_2$};
  \node[state]         (D) [below left of=C] {$q_3$};
  \node[state]         (H) [left of=D] {$h$};

  \path (A) edge node {\TMtrans{\TMblank}{\TMstroke}{\TMstay}} (B)
           edge [loop below] node {\TMtrans{\TMstroke}{\TMstroke}{\TMright}} (B)
        (B) edge [loop below] node {\TMtrans{\TMstroke}{\TMstroke}{\TMright}} (B)
            edge node {\TMtrans{\TMblank}{\TMblank}{\TMleft}} (C)
        (C) edge node {\TMtrans{\TMstroke}{\TMblank}{\TMleft}} (D)
        (D) edge [loop above] node {\TMtrans{\TMstroke}{\TMstroke}{\TMleft}} (D)
        edge node {\TMtrans{\TMendtape}{\TMendtape}{\TMright}} (H);
\end{tikzpicture}
\]\caption{A disciplined addition machine}
\ollabel{fig:adder-disc}
\end{figure}
\end{ex}

\begin{prop}\ollabel{prop:disciplined} For every Turing machine~$M$,
there is a disciplined Turing machine~$M'$ which halts with output~$O$
if $M$~halts with output~$O$, and does not halt if $M$~does not halt.
In particular, and function $f\colon\Nat^n \to \Nat$ computable by a
Turing machine is computable by a disciplined Turing machine.
\end{prop}

\begin{prob}\label{tur:mac:dis:prob:disc-succ}
Give a disciplined machine that computes $f(x) = x+1$.
\end{prob}


\begin{prob}\label{tur:mac:dis:prob:copier}
Find a disciplined machine which, when started on input $\TMstroke^n$
produces output $\TMstroke^n \concat \TMblank \concat \TMstroke^n$.
\end{prob}

\end{document}
