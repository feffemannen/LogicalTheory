% Part: turing-machines
% Chapter: machines-computations
% Section: turing-machines

\documentclass[../../../include/open-logic-section]{subfiles}

\begin{document}

\olfileid{tur}{mac}{tur}
\olsection{Turing Machines}

\begin{explain}
The formal definition of what constitutes a Turing machine looks
abstract, but is actually simple: it merely packs into one
mathematical structure all the information needed to specify the
workings of a Turing machine. This includes (1) which states the
machine can be in, (2) which symbols are allowed to be on the tape, (3)
which state the machine should start in, and (4) what the instruction
set of the machine is.
\end{explain}

\begin{defn}[Turing machine]
A \emph{Turing machine} $M$ is a tuple $\langle Q, \Sigma, q_0,
\delta\rangle$ consisting of
\begin{enumerate}
\item a finite set of \emph{states}~$Q$,
\item a finite \emph{alphabet} $\Sigma$ which includes $\TMendtape$ and
  $\TMblank$,
\item an \emph{initial state}~$q_0 \in Q$,
\item a finite \emph{instruction set}~$\delta\colon Q \times \Sigma
  \pto Q \times \Sigma \times \{\TMleft, \TMright, \TMstay\}$.
\end{enumerate}
The partial function~$\delta$ is also called the \emph{transition function}
of~$M$.
\end{defn}

\begin{explain}
We assume that the tape is infinite in one direction only. For this
reason it is useful to designate a special symbol~$\TMendtape$ as
a marker for the left end of the tape. This makes it easier for
Turing machine programs to tell when they're ``in danger'' of running
off the tape. We will also assume that this symbol is never overwritten,
i.e., that $\delta(q,\TMendtape) = \tuple{q', \TMendtape, x}$ if
$\delta(q,\TMendtape)$ is defined.
\end{explain}

\begin{ex}
\emph{Even Machine:} The even machine is formally
the quadruple $\tuple{Q, \Sigma, q_0, \delta}$ where
\begin{align*}
Q & = \{ q_0, q_1 \} \\
\Sigma & = \{ \TMendtape, \TMblank, \TMstroke \}, \\
\delta(q_0, \TMstroke) & = \tuple{q_1, \TMstroke, \TMright},\\
\delta(q_1, \TMstroke) & = \tuple{q_0, \TMstroke, \TMright},\\
\delta(q_1, \TMblank)  & = \tuple{q_1, \TMblank, \TMright}.
\end{align*}
\end{ex}

\end{document}
