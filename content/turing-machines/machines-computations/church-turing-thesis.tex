% Part: computability
% Chapter: tm-computations
% Section: church-turing-thesis

\documentclass[../../../include/open-logic-section]{subfiles}

\begin{document}

\olfileid{tur}{mac}{ctt}
\olsection{The Church--Turing Thesis}

Turing machines are supposed to be a precise replacement for the
concept of an effective procedure. Turing thought that anyone who
grasped both the concept of an effective procedure and the concept
of a Turing machine would have the intuition that anything that could
be done via an effective procedure could be done by Turing machine.
This claim is given support by the fact that all the other proposed
precise replacements for the concept of an effective procedure turn
out to be extensionally equivalent to the concept of a Turing machine
---that is, they can compute exactly the same set of functions. This
claim is called the \emph{Church--Turing thesis}.

\begin{defn}[Church--Turing thesis]
The \emph{Church--Turing Thesis} states that anything computable via
an effective procedure is Turing computable.
\end{defn}

The Church--Turing thesis is appealed to in two ways.  The first kind
of use of the Church--Turing thesis is an excuse for laziness.  Suppose
we have a description of an effective procedure to compute something,
say, in ``pseudo-code.''  Then we can invoke the Church--Turing thesis
to justify the claim that the same function is computed by some Turing
machine, even if we have not in fact constructed it.

The other use of the Church--Turing thesis is more philosophically
interesting.  It can be shown that there are functions which cannot be
computed by Turing machines.  From this, using the Church--Turing
thesis, one can conclude that it cannot be effectively computed, using
any procedure whatsoever.  For if there were such a procedure, by the
Church--Turing thesis, it would follow that there would be a Turing
machine for it.  So if we can prove that there is no Turing machine
that computes it, there also can't be an effective procedure.  In
particular, the Church--Turing thesis is invoked to claim that the
so-called halting problem not only cannot be solved by Turing
machines, it cannot be effectively solved at all.

\end{document}
