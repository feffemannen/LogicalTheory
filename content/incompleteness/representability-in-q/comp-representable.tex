% Part: incompleteness
% Chapter: representability-in-q
% Section: comp-representable

\documentclass[../../../include/open-logic-section]{subfiles}

\begin{document}

\olfileid{inc}{req}{crq}
\olsection{Computable Functions are Representable in $\Th{Q}$}

\begin{thm}
Every computable function is representable in~$\Th{Q}$.
\end{thm}

\begin{proof}
For definiteness, and using the Church-Turing Thesis, let's say that a
function is computable iff it is general recursive. The general
recursive functions are those which can be defined from the zero
function~$\Zero$, the successor function~$\Succ$, and the projection
function~$\Proj{n}{i}$ using composition, primitive recursion, and
regular minimization. By \olref[pri]{lem:prim-rec}, any function~$h$
that can be defined from $f$ and~$g$ can also be defined using
composition and regular minimization from $f$, $g$, and $\Zero$,
$\Succ$, $\Proj{n}{i}$, $\Add$, $\Mult$, $\Char{=}$. Consequently, a
function is general recursive iff it can be defined from $\Zero$,
$\Succ$, $\Proj{n}{i}$, $\Add$, $\Mult$, $\Char{=}$ using composition
and regular minimization.

We've furthermore shown that the basic functions in question are
representable in~$\Th{Q}$
(\cref{inc:req:bre:prop:rep-zero,inc:req:bre:prop:rep-succ,inc:req:bre:prop:rep-proj,inc:req:bre:prop:rep-id,inc:req:bre:prop:rep-add,inc:req:bre:prop:rep-mult}),
and that any function defined from representable functions by
composition or regular minimization
(\olref[cmp]{prop:rep-composition},
\olref[min]{prop:rep-minimization}) is also representable. Thus every
general recursive function is representable in~$\Th{Q}$.
\end{proof}

\begin{explain}
We have shown that the set of computable functions can be
characterized as the set of functions representable in $\Th{Q}$. In
fact, the proof is more general. From the definition of
representability, it is not hard to see that any theory extending
$\Th{Q}$ (or in which one can interpret $\Th{Q}$) can represent the
computable functions. But, conversely, in any !!{derivation} system in
which the notion of !!{derivation} is computable, every representable
function is computable. So, for example, the set of computable
functions can be characterized as the set of functions representable
in Peano arithmetic, or even Zermelo-Fraenkel set theory. As G\"odel
noted, this is somewhat surprising.  We will see that when it comes to
provability, questions are very sensitive to which theory you
consider; roughly, the stronger the axioms, the more you can prove.
But across a wide range of axiomatic theories, the representable
functions are exactly the computable ones; stronger theories do not
represent more functions as long as they are axiomatizable.
\end{explain}

\end{document}
