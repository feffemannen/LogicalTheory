% Part: lambda-calculus
% Chapter: lambda-definability
% Section: introduction

\documentclass[../../../include/open-logic-section]{subfiles}

\begin{document}

\olfileid{lam}{ldf}{int}
\olsection{Introduction}

At first glance, the lambda calculus is just a very abstract calculus
of expressions that represent functions and applications of them to
others. Nothing in the syntax of the lambda calculus suggests that
these are functions of particular kinds of objects, in particular, the
syntax includes no mention of natural numbers. Its basic
operations---application and lambda abstractions---are operations that
apply to any function, not just functions on natural
numbers.

Nevertheless, with some ingenuity, it is possible to define
arithmetical functions, i.e., functions on the natural numbers, in the
lambda calculus. To do this, we define, for each natural number~$n \in
\Nat$, a special $\lambd$-term~$\num n$, the \emph{Church numeral}
for~$n$. (Church numerals are named for Alonzo Church.)

\begin{defn}
  If $n \in \Nat$, the corresponding \emph{Church numeral} $\num{n}$
  represents~$n$:
  \[
    \num{n} \ident \lambd[fx][f^n(x)]
  \]
  Here, $f^n(x)$ stands for the result of applying $f$ to~$x$ $n$
  times. For example, $\num{0}$ is $\lambd[fx][x]$, and $\num{3}$ is
  $\lambd[fx][f(f(f\,x))]$.
\end{defn}
  
The Church numeral $\num n$ is encoded as a lambda term which
represents a function accepting two arguments $f$ and $x$, and
returns $f^n(x)$. Church numerals are evidently in normal form.

A represention of natural numbers in the lambda calculus is only
useful, of course, if we can compute with them.  Computing with Church
numerals in the lambda calculus means applying a $\lambd$-term~$F$ to
such a Church numeral, and reducing the combined term~$F\, \num n$ to
a normal form. If it always reduces to a normal form, and the normal
form is always a Church numeral~$\num m$, we can think of the output
of the computation as being the number~$m$. We can then think of~$F$
as defining a function $f\colon \Nat \to \Nat$, namely the function
such that $f(n) = m$ iff $F\, \num n \red \num m$. Because of the
Church-Rosser property, normal forms are unique if they exist. So if
$F\, \num n \red \num m$, there can be no other term in normal form,
in particular no other Church numeral, that $F \, \num n$ reduces to.

Conversely, given a function $f\colon \Nat \to \Nat$,
we can ask if there is a term $F$ that defines~$f$ in this way. In
that case we say that $F$ \emph{!!{lambda define}s}~$f$, and that $f$ is
!!{lambda definable}. We can generalize this to many-place and partial
functions.

\begin{defn}
Suppose $f\colon \Nat^k \to \Nat$. We say that a lambda term~$F$
\emph{!!{lambda define}s} $f$ if for all 
$n_0$, \dots,~$n_{k-1}$,
\[
F \, \num{n_0} \, \num{n_1} \dots \num{n_{k-1}} \red \num{f(n_0, n_1, \dots,
  n_{k-1})}
\]
if $f(n_0, \dots, n_{k-1})$ is defined, and $F \, \num{n_0} \,
\num{n_1} \dots \num{n_{k-1}}$ has no normal form otherwise.
\end{defn}

A very simple example are the constant functions. The term $C_k \ident
\lambd[x][\num{k}]$ !!{lambda define}s the function $c_k\colon \Nat \to
\Nat$ such that $c(n) = k$. For $C_k \, \num n \ident
(\lambd[x][\num{k}])\num n \redone \num{k}$ for any~$n$. The identity
function is !!{lambda defined} by $\lambd[x][x]$. More complex
functions are of course harder to define, and often require a lot of
ingenuity. So it is perhaps surprising that every computable function
is !!{lambda definable}. The converse is also true: if a function is
!!{lambda definable}, it is computable.

\end{document}
