% Part: lambda-calculus
% Chapter: representability
% Section: currying

\documentclass[../../../include/open-logic-section]{subfiles}

\begin{document}

\olfileid{lam}{rep}{cur} 
\olsection{Currying}

A $\lambd$-abstract $\lambd[x][M]$ represents a function of one
argument, which is quite a limitation when we want to define function
accepting multiple arguments.  One way to do this would be by
extending the $\lambd$-calculus to allow the formation of pairs,
triples, etc., in which case, say, a three-place function
$\lambd[x][M]$ would expect its argument to be a triple.  However, it
is more convenient to do this by \emph{Currying}.

Let's consider an example. We'll pretend for a moment that we have a
$+$ operation in the $\lambd$-calculus. The addition function is
$2$-place, i.e., it takes two arguments.  But a $\lambd$-abstract only
gives us functions of one argument: the syntax does not allow
expressions like $\lambd[(x,y)][(x+y)]$.  However, we can consider the
one-place function~$f_x(y)$ given by $\lambd[y][(x+y)]$, which adds
$x$ to its single argument~$y$.  Actually, this is not a single
function, but a family of different functions ``add $x$,'' one for
each number~$x$.  Now we can define another one-place function~$g$ as
$\lambd[x][f_x]$. Applied to argument $x$, $g(x)$ returns the
function~$f_x$---so its values are other functions.  Now if we apply
$g$ to $x$, and then the result to~$y$ we get: $(g(x))y = f_x(y) =
x+y$.  In this way, the one-place function~$g$ can do the same job as
the two-place addition function. ``Currying'' simply refers to this
trick for turning two-place functions into one place functions (whose
values are one-place functions).

Here is an example properly in the syntax of the $\lambd$-calculus.
How do we represent the function $f(x,y) = x$? If we want to define a
function that accepts two arguments and returns the first, we can
write $\lambd[x][\lambd[y][x]]$, which literally is a function that
accepts an argument~$x$ and returns the function~$\lambd[y][x]$. The
function $\lambd[y][x]$ accepts another argument~$y$, but drops it,
and always returns~$x$.  Let's see what happens when we apply
$\lambd[x][\lambd[y][x]]$ to two arguments:
\begin{align*}
  (\lambd[x][\lambd[y][x]])MN 
  \bredone & (\lambd[y][M])N \\
  \bredone & M
\end{align*}

In general, to write a function with parameters $x_1$, \dots,~$x_n$
defined by some term~$N$, we can write
$\lambd[x_1][\lambd[x_2][\ldots\lambd[x_n][N]]]$. If we apply $n$ arguments
to it we get:
\begin{multline*}
  (\lambd[x_1][\lambd[x_2][\ldots\lambd[x_n][N]]]) M_1 \dots M_n \bredone\\
  \begin{aligned}
  \bredone {} & (\Subst{(\lambd[x_2][\ldots\lambd[x_n][N]])}{M_1}{x_1}) M_2
  \dots M_n\\
   \eqs {} & (\lambd[x_2][\ldots\lambd[x_n][\Subst{N}{M_1}{x_1}]]) M_2
            \dots M_n \\
  \vdots & \\
  \bredone {} &\Subst{\Subst{P}{M_1}{x_1}\ldots}{M_n}{x_n}
  \end{aligned}
\end{multline*}
The last line literally means substituting $M_i$ for $x_i$ in the body
of the function definition, which is exactly what we want when
applying multiple arguments to a function.
\end{document}

