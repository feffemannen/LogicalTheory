% part: intuitionistic-logic
% chapter: semantics
% section: topological-semantics

\documentclass[../../../include/open-logic-section]{subfiles}

\begin{document}

\olfileid{int}{sem}{top}

\olsection{Topological Semantics}

Another way to provide a semantics for intuitionistic logic is using
the mathematical concept of a topology.

\begin{defn}
  Let $X$ be a set. A \emph{topology on~$X$} is a set $\Top{O}
  \subseteq \Pow{X}$ that satisfies the properties below. The
  !!{element}s of~$\Top{O}$ are called the \emph{open sets} of the
  topology. The set $X$ together with $\Top{O}$ is called a
  \emph{topological space}.
  \begin{enumerate}
  \item The empty set and the entire space are open: $\emptyset$, $X \in
    \Top{O}$.
  \item Open sets are closed under finite intersections: if $U$, $V \in
    \Top{O}$ then $U \cap V \in \Top{O}$
  \item Open sets are closed under arbitrary unions: if $U_i \in
    \Top{O}$ for all $i \in I$, then $\bigcup \Setabs{U_i}{i \in
      I} \in \Top{O}$.
  \end{enumerate}
\end{defn}

We may write $X$ for a topology if the collection of open sets can be
inferred from the context; note that, still, only after $X$ is endowed
with open sets can it be called a topology.

\begin{defn}
  A \emph{topological model} of intuitionistic propositional logic is
  a triple $\mModel{X} = \tuple{X, \Top{O}, V}$ where $\Top{O}$ is a
  topology on~$X$ and $V$ is a function assigning an open set in
  $\Top{O}$ to each propositional variable.

  Given a topological model~$\mModel{X}$, we can define $\Prop{X}{!A}$
  inductively as follows:
  \begin{enumerate}
  \item $\Prop{X}{\lfalse} = \emptyset$
  \item $\Prop{X}{p} = V(p)$
  \item $\Prop{X}{!A \land !B} = \Prop{X}{!A} \cap \Prop{X}{!B}$
  \item $\Prop{X}{!A \lor !B} = \Prop{X}{!A} \cup \Prop{X}{!B}$
  \item $\Prop{X}{!A \lif !B} = \Interior{(X \setminus \Prop{X}{!A}) \cup
    \Prop{X}{!B}}$
  \end{enumerate}
  Here, $\Interior{V}$ is the function that maps a set $V \subseteq X$
  to its \emph{interior}, that is, the union of all open sets it
  contains. In other words,
  \[
  \Interior{V} = \bigcup \Setabs{U}{U \subseteq V \text{ and } U \in \Top{O}}.
  \]
\end{defn}

Note that the interior of any set is always open, since it is a union
of open sets. Thus, $\Prop{X}{!A}$ is always an open set.

Although topological semantics is highly abstract, there are ways to
think about it that might motivate it. Suppose that the !!{element}s,
or ``points,'' of $X$ are points at which statements can be
evaluated. The set of all points where $!A$ is true is the proposition
expressed by~$!A$. Not every set of points is a potential proposition;
only the !!{element}s of $\Top{O}$ are.  $!A \Entails !B$ iff $!B$ is
true at every point at which~$!A$ is true, i.e., $\Prop{X}{!A}
\subseteq \Prop{X}{!B}$, for all~$X$. The absurd statement~$\lfalse$
is never true, so $\Prop{X}{\lfalse} = \emptyset$. How must the
propositions expressed by $!B \land !C$, $!B \lor !C$, and $!B \lif
!C$ be related to those expressed by $!B$ and~$!C$ for the
intuitionistically valid laws to hold, i.e., so that $!A \Proves !B$
iff $\Prop{X}{!A} \subset \Prop{X}{!B}$. $\lfalse \Proves !A$ for any
$!A$, and only $\emptyset \subseteq U$ for all $U$.  Since $!B \land
!C \Proves !B$, $\Prop{X}{!B \land !C} \subseteq \Prop{X}{!B}$, and
similarly $\Prop{X}{!B \land !C} \subseteq \Prop{X}{!C}$. The largest
set satisfying $W \subseteq U$ and $W \subseteq V$ is $U \cap V$.
Conversely, $!B \Proves !B \lor !C$ and $!C \Proves !B \lor !C$, and
so $\Prop{X}{!B} \subseteq \Prop{X}{!B \lor !C}$ and $\Prop{X}{!C}
\subseteq \Prop{X}{!B \lor !C}$. The smallest set~$W$ such that $U
\subseteq W$ and $V \subseteq W$ is $U \cup V$.  The definition for
$\lif$ is tricky: $!A \lif !B$ expresses the weakest proposition that,
combined with $!A$, entails $!B$. That $!A \lif !B$ combined with $!A$
entails~$!B$ is clear from $(!A \lif !B) \land !A \Proves !B $. So
$\Prop{X}{!A \lif !B}$ should be the greatest open set such that
$\Prop{X}{!A \lif !B} \cap \Prop{X}{!A} \subset \Prop{X}{!B}$, leading
to our definition.

\end{document}
