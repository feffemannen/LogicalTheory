% Part: first-order-logic
% Chapter: completeness
% Section: introduction

\documentclass[../../../include/open-logic-section]{subfiles}

\begin{document}

\iftag{FOL}
      {\olfileid{fol}{com}{int}}
      {\olfileid{pl}{com}{int}}

\olsection{Introduction}

The completeness theorem is one of the most fundamental results about
logic.  It comes in two formulations, the equivalence of which we'll
prove.  In its first formulation it says something fundamental about
the relationship between semantic consequence and our !!{derivation}
system: if !!a{sentence}~$!A$ follows from some !!{sentence}s
$\Gamma$, then there is also !!a{derivation} that establishes $\Gamma
\Proves !A$. Thus, the !!{derivation} system is as strong as it can
possibly be without proving things that don't actually follow.

In its second formulation, it can be stated as a model existence
result: every consistent set of !!{sentence}s is satisfiable.
Consistency is a proof-theoretic notion: it says that our
!!{derivation} system is unable to produce certain !!{derivation}s.
But who's to say that just because there are no !!{derivation}s of a
certain sort from~$\Gamma$, it's guaranteed that there is
\iftag{FOL}{!!a{structure}~$\Struct{M}$}{!!{valuation}{~$\pAssign{v}$}
with $\iftag{FOL}{\Sat{M}}{\pSat{v}}{\Gamma}$}? Before the
completeness theorem was first proved---in fact before we had the
!!{derivation} systems we now do---the great German mathematician
David Hilbert held the view that consistency of mathematical theories
guarantees the existence of the objects they are about. He put it as
follows in a letter to Gottlob Frege:
\begin{quote}
  If the arbitrarily given axioms do not contradict one another with
  all their consequences, then they are true and the things defined by
  the axioms exist. This is for me the criterion of truth and
  existence. 
\end{quote}
Frege vehemently disagreed. The second formulation of the completeness
theorem shows that Hilbert was right in at least the sense that if the
axioms are consistent, then \emph{some}
\iftag{FOL}{!!{structure}}{!!{valuation}} exists that makes them all
true.

These aren't the only reasons the completeness theorem---or rather,
its proof---is important.  It has a number of important consequences,
some of which we'll discuss separately.  For instance, since any
!!{derivation} that shows $\Gamma \Proves !A$ is finite and so can
only use finitely many of the !!{sentence}s in~$\Gamma$, it follows by
the completeness theorem that if $!A$ is a consequence of~$\Gamma$, it
is already a consequence of a finite subset of~$\Gamma$.  This is
called \emph{compactness}.  Equivalently, if every finite subset of
$\Gamma$ is consistent, then $\Gamma$ itself must be consistent.

Although the compactness theorem follows from the completeness theorem
via the detour through !!{derivation}s, it is also possible to use the
\emph{the proof of} the completeness theorem to establish it
directly. For what the proof does is take a set of !!{sentence}s with
a certain property---consistency---and constructs !!a{structure} out
of this set that has certain properties (in this case, that it
satisfies the set). Almost the very same construction can be used to
directly establish compactness, by starting from ``finitely
satisfiable'' sets of !!{sentence}s instead of consistent ones.
\iftag{FOL}{The construction also yields other consequences, e.g.,
  that any satisfiable set of !!{sentence}s has a finite or
  !!{denumerable}s model.  (This result is called the
  L\"owenheim--Skolem theorem.)  In general, the construction of
  !!{structure}s from sets of !!{sentence}s is used often in logic,
  and sometimes even in philosophy.}{}

\end{document}
