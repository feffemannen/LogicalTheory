% Part: first-order-logic
% Chapter: natural-deduction
% Section: proving-things

\documentclass[../../../include/open-logic-section]{subfiles}

\begin{document}

\iftag{FOL}
      {\olfileid{fol}{ntd}{pro}}
      {\olfileid{pl}{ntd}{pro}}

\olsection{Examples of \usetoken{P}{derivation}}

\begin{ex}
Let's give a !!{derivation} of the !!{sentence} $(!A \land !B) \lif !A$.

We begin by writing the desired conclusion at the bottom of the 
!!{derivation}.
\begin{prooftree}
\AxiomC{}
\UnaryInfC{$(!A\land !B) \lif !A$}
\end{prooftree}

Next, we need to figure out what kind of inference could result in
!!a{sentence} of this form. The !!{main operator} of the
conclusion is $\lif$, so we'll try to arrive at the
conclusion using the \Intro{\lif} rule. It is best to write down
the assumptions involved and label the inference rules as you
progress, so it is easy to see whether all assumptions have been
!!{discharged} at the end of the proof.
\begin{prooftree}
\AxiomC{$\Discharge{!A \land !B}{1}$}
\DeduceC{$!A$}
\DischargeRule{\Intro{\lif}}{1} 
\UnaryInfC{$(!A\land !B) \lif !A$}
\end{prooftree}

We now need to fill in the steps from the assumption $!A \land !B$ to $!A$.
Since we only have one connective to deal with, $\land$, we must
use the $\land$ elim rule. This gives us the following proof:
\begin{prooftree}
\AxiomC{$\Discharge{!A \land !B}{1}$}
\RightLabel{\Elim{\land}}
\UnaryInfC{$!A$}
\DischargeRule{\Intro{\lif}}{1} 
\UnaryInfC{$(!A\land !B) \lif !A$}
\end{prooftree}
We now have a correct !!{derivation} of $(!A \land
!B) \lif !A$.
\end{ex}

\begin{ex}
Now let's give a !!{derivation} of $(\lnot !A \lor !B)
\lif (!A \lif !B)$.

We begin by writing the desired conclusion at the bottom of the 
derivation.
\begin{prooftree}
\AxiomC{}
\UnaryInfC{$(\lnot !A \lor !B) \lif (!A \lif !B)$}
\end{prooftree}
To find a logical rule that could give us this conclusion, we
look at the logical connectives in the conclusion: $\lnot$,
$\lor$, and $\lif$. We only care at the moment about the first
occurence of $\lif$ because it is the !!{main operator} of the
!!{sentence} in the end-sequent, while $\lnot$, $\lor$ and the second
occurence of $\lif$ are inside the scope of another connective, so we
will take care of those later. We therefore start with the
\Intro{\lif} rule.  A correct application must look like this:
\begin{prooftree}
\AxiomC{$\Discharge{\lnot !A \lor !B}{1}$}
\DeduceC{$!A \lif !B$}
\DischargeRule{\Intro{\lif}}{1}
\UnaryInfC{$(\lnot !A \lor !B) \lif (!A \lif !B)$}
\end{prooftree}
This leaves us with two possibilities to continue. Either we can
keep working from the bottom up and look for another application
of the \Intro{\lif} rule, or we can work from the top down and apply a
\Elim{\lor} rule. Let us apply the latter. We will use the assumption
$\lnot !A \lor !B$ as the leftmost premise of \Elim{\lor}.  For a valid
application of \Elim{\lor}, the other two premises must be identical
to the conclusion $!A \lif !B$, but each may be derived in turn from
another assumption, namely the two disjuncts of $\lnot !A \lor !B$.
So our !!{derivation} will look like this:
\begin{prooftree}
\AxiomC{$\Discharge{\lnot !A \lor !B}{1}$}
\AxiomC{$\Discharge{\lnot !A}{2}$}
\DeduceC{$!A \lif !B$}
\AxiomC{$\Discharge{!B}{2}$}
\DeduceC{$!A \lif !B$}
\DischargeRule{\Elim{\lor}}{2}
\TrinaryInfC{$!A \lif !B$}
\DischargeRule{\Intro{\lif}}{1} 
\UnaryInfC{$(\lnot !A \lor !B) \lif (!A \lif !B)$}
\end{prooftree}

In each of the two branches on the right, we want to !!{derive} $!A
\lif !B$, which is best done using \Intro{\lif}.
\begin{prooftree}
\AxiomC{$\Discharge{\lnot !A \lor !B}{1}$}
\AxiomC{$\Discharge{\lnot !A}{2}, \Discharge{!A}{3}$}
\DeduceC{$!B$}
\DischargeRule{\Intro{\lif}}{3}
\UnaryInfC{$!A \lif !B$}
\AxiomC{$\Discharge{!B}{2}, \Discharge{!A}{4}$}
\DeduceC{$!B$}
\DischargeRule{\Intro{\lif}}{4}
\UnaryInfC{$!A \lif !B$}
\DischargeRule{\Elim{\lor}}{2}
\TrinaryInfC{$!A \lif !B$}
\DischargeRule{\Intro{\lif}}{1} 
\UnaryInfC{$(\lnot !A \lor !B) \lif (!A \lif !B)$}
\end{prooftree}

For the two missing parts of the !!{derivation}, we need
!!{derivation}s of $!B$ from $\lnot !A$ and $!A$ in the middle, and
from $!A$ and $!B$ on the left.  Let's take the former first. $\lnot
!A$ and $!A$ are the two premises of \Elim{\lnot}:
\begin{prooftree}
\AxiomC{$\Discharge{\lnot !A}{2}$}
\AxiomC{$\Discharge{!A}{3}$}
\RightLabel{\Elim{\lnot}}
\BinaryInfC{$\lfalse$}
\DeduceC{$!B$}
\end{prooftree}
By using \FalseInt, we can obtain $!B$ as a conclusion and
complete the branch.
\begin{prooftree}
\AxiomC{$\Discharge{\lnot !A \lor !B}{1}$}
\AxiomC{$\Discharge{\lnot !A}{2}$}
\AxiomC{$\Discharge{!A}{3}$}
\RightLabel{\Intro{\lfalse}}
\BinaryInfC{$\lfalse$}
\RightLabel{\FalseInt}
\UnaryInfC{$!B$}
\DischargeRule{\Intro{\lif}}{3}
\UnaryInfC{$!A \lif !B$}
\AxiomC{$\Discharge{!B}{2}, \Discharge{!A}{4}$}
\DeduceC{$!B$}
\DischargeRule{\Intro{\lif}}{4}
\UnaryInfC{$!A \lif !B$}
\DischargeRule{\Elim{\lor}}{2}
\TrinaryInfC{$!A \lif !B$}
\DischargeRule{\Intro{\lif}}{1} 
\UnaryInfC{$(\lnot !A \lor !B) \lif (!A \lif !B)$}
\end{prooftree}

Let's now look at the rightmost branch.  Here it's important to
realize that the definition of !!{derivation} \emph{allows assumptions
  to be discharged} but \emph{does not require} them to be.  In other
words, if we can derive $!B$ from one of the assumptions $!A$ and $!B$
without using the other, that's ok.  And to !!{derive} $!B$ from~$!B$
is trivial: $!B$ by itself is such a !!{derivation}, and no inferences
are needed.  So we can simply delete the assumption~$!A$.
\begin{prooftree}
\AxiomC{$\Discharge{\lnot !A \lor !B}{1}$}
\AxiomC{$\Discharge{\lnot !A}{2}$}
\AxiomC{$\Discharge{!A}{3}$}
\RightLabel{\Elim{\lnot}}
\BinaryInfC{$\lfalse$}
\RightLabel{\FalseInt}
\UnaryInfC{$!B$}
\DischargeRule{\Intro{\lif}}{3}
\UnaryInfC{$!A \lif !B$}
\AxiomC{$\Discharge{!B}{2}$}
\RightLabel{\Intro{\lif}}
\UnaryInfC{$!A \lif !B$}
\DischargeRule{\Elim{\lor}}{2}
\TrinaryInfC{$!A \lif !B$}
\DischargeRule{\Intro{\lif}}{1} 
\UnaryInfC{$(\lnot !A \lor !B) \lif (!A \lif !B)$}
\end{prooftree}
Note that in the finished !!{derivation}, the rightmost \Intro{\lif}
inference does not actually discharge any assumptions.
\end{ex}

\begin{ex}
So far we have not needed the \FalseCl{} rule. It is special in that
it allows us to discharge an assumption that isn't a sub-!!{formula} of
the conclusion of the rule.  It is closely related to the \FalseInt{}
rule. In fact, the \FalseInt{} rule is a special case of the
\FalseCl{} rule---there is a logic called ``intuitionistic logic'' in
which only \FalseInt{} is allowed.  The \FalseCl{} rule is a last
resort when nothing else works.  For instance, suppose we want to
!!{derive} $!A \lor \lnot !A$. Our usual strategy would be to attempt
to !!{derive} $!A \lor \lnot !A$ using $\Intro{\lor}$. But this would
require us to !!{derive} either $!A$ or $\lnot !A$ from no
assumptions, and this can't be done. \FalseCl{} to the rescue!
\begin{prooftree}
  \AxiomC{$\Discharge{\lnot(!A \lor \lnot !A)}{1}$}
  \DeduceC{$\lfalse$}
  \DischargeRule{\FalseCl}{1}
  \UnaryInfC{$!A \lor \lnot !A$}
\end{prooftree}
Now we're looking for !!a{derivation} of $\lfalse$ from $\lnot(!A \lor
\lnot !A)$. Since $\lfalse$ is the conclusion of $\Elim{\lnot}$ we
might try that:
\begin{prooftree}
  \AxiomC{$\Discharge{\lnot(!A \lor \lnot !A)}{1}$}
  \DeduceC{$\lnot !A$}
  \AxiomC{$\Discharge{\lnot(!A \lor \lnot !A)}{1}$}
  \DeduceC{$!A$}
  \RightLabel{\Elim{\lnot}}
  \BinaryInfC{$\lfalse$}
  \DischargeRule{\FalseCl}{1}
  \UnaryInfC{$!A \lor \lnot !A$}
\end{prooftree}
Our strategy for finding a !!{derivation} of~$\lnot !A$ calls for an
application of~$\Intro{\lnot}$:
\begin{prooftree}
  \AxiomC{$\Discharge{\lnot(!A \lor \lnot !A)}{1}, \Discharge{!A}{2}$}
  \DeduceC{$\lfalse$}
  \DischargeRule{\Intro{\lnot}}{2}
  \UnaryInfC{$\lnot !A$}
  \AxiomC{$\Discharge{\lnot(!A \lor \lnot !A)}{1}$}
  \DeduceC{$!A$}
  \RightLabel{\Elim{\lnot}}
  \BinaryInfC{$\lfalse$}
  \DischargeRule{\FalseCl}{1}
  \UnaryInfC{$!A \lor \lnot !A$}
\end{prooftree}
Here, we can get $\lfalse$ easily by applying $\Elim{\lnot}$ to the
assumption $\lnot(!A \lor \lnot !A)$ and $!A \lor \lnot !A$ which
follows from our new assumption $!A$ by~$\Intro{\lor}$:
\begin{prooftree}
  \AxiomC{$\Discharge{\lnot(!A \lor \lnot !A)}{1}$}
  \AxiomC{$\Discharge{!A}{2}$}
  \RightLabel{\Intro{\lor}}
  \UnaryInfC{$!A \lor \lnot !A$}
  \RightLabel{\Elim{\lnot}}
  \BinaryInfC{$\lfalse$}
  \DischargeRule{\Intro{\lnot}}{2}
  \UnaryInfC{$\lnot !A$}
  \AxiomC{$\Discharge{\lnot(!A \lor \lnot !A)}{1}$}
  \DeduceC{$!A$}
  \RightLabel{\Elim{\lnot}}
  \BinaryInfC{$\lfalse$}
  \DischargeRule{\FalseCl}{1}
  \UnaryInfC{$!A \lor \lnot !A$}
\end{prooftree}
On the right side we use the same strategy, except we get $!A$ by~\FalseCl:
\begin{prooftree}
  \AxiomC{$\Discharge{\lnot(!A \lor \lnot !A)}{1}$}
  \AxiomC{$\Discharge{!A}{2}$}
  \RightLabel{\Intro{\lor}}
  \UnaryInfC{$!A \lor \lnot !A$}
  \RightLabel{\Elim{\lnot}}
  \BinaryInfC{$\lfalse$}
  \DischargeRule{\Intro{\lnot}}{2}
  \UnaryInfC{$\lnot !A$}
  \AxiomC{$\Discharge{\lnot(!A \lor \lnot !A)}{1}$}
  \AxiomC{$\Discharge{\lnot !A}{3}$}
  \RightLabel{\Intro{\lor}}
  \UnaryInfC{$!A \lor \lnot !A$}
  \RightLabel{\Elim{\lnot}}
  \BinaryInfC{$\lfalse$}
  \DischargeRule{\FalseCl}{3}
  \UnaryInfC{$!A$}
  \RightLabel{\Elim{\lnot}}
  \BinaryInfC{$\lfalse$}
  \DischargeRule{\FalseCl}{1}
  \UnaryInfC{$!A \lor \lnot !A$}
\end{prooftree}
\end{ex}

\begin{prob}
Give !!{derivation}s of the following:
\begin{enumerate}
\item $\lnot(!A \lif !B) \lif (!A \land \lnot !B)$
\item $(!A \lif !C) \lor (!B \lif !C)$ from the assumption $(!A \land
  !B) \lif !C$
\item \( \lnot \lnot !A \lif !A \)
\item \( \lnot !A \lif \lnot !B \) from the assumption \( !B \lif !A \)
\item \( \lnot !A \) from the assumption \( ( !A \lif \lnot !A ) \)
\item \( !A \) from the assumptions \( !B \lif !A \) and \( \lnot !B \lif !A \)
\end{enumerate}
\end{prob}

\end{document}
