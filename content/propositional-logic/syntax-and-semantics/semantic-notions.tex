% Part: propositional-logic
% Chapter: syntax-and-semantics
% Section: semantic-notions

\documentclass[../../../include/open-logic-section]{subfiles}

\begin{document}

\olfileid{pl}{syn}{sem}

\olsection{Semantic Notions}

We define the following semantic notions:

\begin{defn} 
\begin{enumerate}
\item !!^a{formula}~$!A$ is \emph{satisfiable} if for
  some~$\pAssign{v}$, $\pSat{v}{!A}$; it is
  \emph{unsatisfiable} if for no $\pAssign{v}$, $\pSat{v}{!A}$;
\item !!^a{formula}~$!A$ is a \emph{tautology} if $\pSat{v}{!A}$ for
  all !!{valuation}s~$v$;
\item !!^a{formula}~$!A$ is \emph{contingent} if it is satisfiable but
  not a tautology;
\item If $\Gamma$ is a set of !!{formula}s, $\Gamma \Entails !A$ (``$\Gamma$
  entails $!A$'') if and only if $\pSat{v}{!A}$ for every
  !!{valuation}~$\pAssign{v}$ for which $\pSat{v}{\Gamma}$.
\item If $\Gamma$ is a set of !!{formula}s, $\Gamma$ is
  \emph{satisfiable} if there is !!a{valuation}~$\pAssign{v}$ for which
  $\pSat{v}{\Gamma}$, and $\Gamma$ is
  \emph{unsatisfiable} otherwise.
\end{enumerate} 
\end{defn}

\begin{prob} 
 For each of the following four !!{formula}s determine whether it
 is (a)~satisfiable, (b)~tautology, and (c)~contingent.
\begin{enumerate}
  \item \( ( \Obj p_0 \lif ( \lnot \Obj p_1 \lif \lnot \Obj p_0 ) ) \).
  \item \( ( ( \Obj p_0 \land \lnot \Obj p_1 ) \lif ( \lnot \Obj p_0 \land \Obj p_2 )) \liff ( ( \Obj p_2 \lif \Obj p_0 ) \lif ( \Obj p_0 \lif \Obj p_1 )) \).
  \item \( ( \Obj p_0 \liff \Obj p_1 ) \lif ( \Obj p_2 \liff \lnot \Obj p_1 ) \).
  \item \( (( \Obj p_0 \liff ( \lnot \Obj p_1 \land \Obj p_2 )) \lor ( \Obj p_2 \lif ( \Obj p_0 \liff \Obj p_1 ))) \).
\end{enumerate}
\end{prob}

\begin{prop}
\ollabel{prop:semanticalfacts} 
\begin{enumerate} 
\item $!A$ is a tautology if and only if
  $\emptyset \Entails !A$; 
\item If $\Gamma \Entails !A$ and $\Gamma \Entails !A \lif !B$ then
  $\Gamma \Entails !B$;
\item If $\Gamma$ is satisfiable then every finite subset of $\Gamma$
  is also satisfiable; 
\item \ollabel{def:Monotony} Monotony: if $\Gamma \subseteq \Delta$
  and $\Gamma \Entails !A$ then also $\Delta \Entails !A$;
\item \ollabel{def:Cut} Transitivity: if $\Gamma \Entails !A$ and
  $\Delta \cup \{ !A\} \Entails !B$ then $\Gamma \cup \Delta \Entails
  !B$.
\end{enumerate}
\end{prop}

\begin{proof}
Exercise.
\end{proof}

\begin{prob}
Prove \olref[pl][syn][sem]{prop:semanticalfacts}
\end{prob}

\begin{prop}\ollabel{prop:entails-unsat}
  $\Gamma \Entails !A$ if and only if $\Gamma \cup \{\lnot !A\}$
  is unsatisfiable.
\end{prop}

\begin{proof}
Exercise.
\end{proof}

\begin{prob}
Prove \olref[pl][syn][sem]{prop:entails-unsat}
\end{prob}

\begin{thm}[Semantic Deduction Theorem]
  \ollabel{thm:sem-deduction} $\Gamma \Entails !A \lif !B$ if and only
  if $\Gamma \cup \{!A\} \Entails !B$.
\end{thm}

\begin{proof}
Exercise.
\end{proof}

\begin{prob}
Prove \olref[pl][syn][sem]{thm:sem-deduction}
\end{prob}

We write $!A \Entails !B$ for $\Gamma \Entails !B$ when $\Gamma = \{!A\}$ is a singleton and say that two !!{formula}s are semantically equivalent, $!A \approx !B$, when $!A \Entails !B$ and $!B \Entails !A$, i.e., when $\pValue{v}(!A) = \pValue{v}(!B)$ for all valuations $\pAssign{v}$.

\begin{prop}[Substitution Lemma]\ollabel{prop:substitution-lemma}
  If \( !A_1 \approx !B_1 \), \dots, \( !A_n \approx !B_n \), then \( \SSubst{!A}{\subst{!A_1}{\Obj p_1},\dots,\subst{!A_n}{\Obj p_n}} \approx \SSubst{!A}{\subst{!B_1}{\Obj p_1},\dots,\subst{!B_n}{\Obj p_n}} \).
\end{prop}

\begin{proof}
  Exercise.
\end{proof}

The following equivalences, known as the De Morgan laws, seem to indicate that the connectives $\land$ and $\lor$ behave in a similar, dual, way. 
  \begin{align*}
    ({!A} \land {!B}) &\approx \lnot(\lnot{!A} \lor \lnot{!B})\\
    ({!A} \lor {!B}) &\approx \lnot(\lnot{!A} \land \lnot{!B})
  \end{align*}
This symmetry, or duality, between conjunction and disjunction can be made precise, but first we define the dual of !!a{formula}.

\begin{defn}
The mapping that maps !!a{formula} with no occurrences of $\lif$ nor $\liff$ to its \emph{dual} is defined by the following clauses:
\begin{itemize}
  \item $!A^d \ident !A$ when $!A$ is atomic,
  \item $(\lnot !A )^d \ident \lnot !A^d$,
  \item $(!A \land !B)^d \ident !A^d \lor !B^d$,
  \item $(!A \lor !B)^d \ident !A^d \land !B^d$.
\end{itemize}
\end{defn}

Observe that the dual of the dual of !!a{formula} is the !!{formula} itself, i.e., that $(!A^d)^d \ident !A$.

\begin{prop}\ollabel{thm:dual}
$!A \approx !B$ iff $!A^d \approx !B^d$ whenever the dual is defined.
\end{prop}
\begin{proof}
Exercise.
\end{proof}

\begin{prob}
Prove \olref[pl][syn][sem]{thm:dual} by introducing an auxiliary mapping $!A^n$ just as $!A^d$ except for atomic !!{formula}s where $!A^n$ is defined to be $\lnot !A$ and proving that $!A^n \approx \lnot !A$.
\end{prob}

\end{document}
