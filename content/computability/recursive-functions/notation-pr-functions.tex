% Part: computability
% Chapter: recursive-functions
% Section: notations-pr-functions

\documentclass[../../../include/open-logic-section]{subfiles}

\begin{document}

\olfileid{cmp}{rec}{not}
\olsection{Primitive Recursion Notations}

One advantage to having the precise inductive description of the primitive
recursive functions is that we can be systematic in describing them.
For example, we can assign a ``notation'' to each such function, as
follows. Use symbols $\Zero$, $\Succ$, and $\Proj{n}{i}$ for zero,
successor, and the projections. Now suppose $h$ is defined by
composition from a $k$-place function~$f$ and $n$-place functions $g_0$,
\dots,~$g_{k-1}$, and we have assigned notations $F$, $G_0$,
\dots,~$G_{k-1}$ to the latter functions. Then, using a new symbol
$\fn{Comp}_{k,n}$, we can denote the function $h$ by
$\fn{Comp}_{k,n}[F,G_0,\dots,G_{k-1}]$. 

For functions defined by primitive recursion, we can use analogous
notations. Suppose the $(k+1)$-ary function~$h$ is defined by
primitive recursion from the $k$-ary function~$f$ and the $(k+2)$-ary
function~$g$, and the notations assigned to $f$ and~$g$ are $F$
and~$G$, respectively. Then the notation assigned to~$h$ is
$\fn{Rec}_k[F,G]$. 

Recall that the addition function is defined by primitive recursion as
\begin{align*}
  \Add(x_0, 0) & = \Proj{1}{0}(x_0) = x_0\\
  \Add(x_0, y+1) & = \Succ(\Proj{3}{2}(x_0, y, \Add(x_0, y))) = \Add(x_0, y) +1
\end{align*}
Here the role of~$f$ is played by $\Proj{1}{0}$, and the role of~$g$
is played by $\Succ(\Proj{3}{2}(x_0, y, z))$, which is assigned the
notation $\fn{Comp}_{1,3}[\Succ,\Proj{3}{2}]$ as it is the result of
defining a function by composition from the $1$-ary function~$\Succ$
and the $3$-ary function~$\Proj{3}{2}$. With this setup, we can denote
the addition function by
\[
\fn{Rec}_1[\Proj{1}{0},\fn{Comp}_{1,3}[\Succ,\Proj{3}{2}]].
\]
Having these notations sometimes proves useful, e.g., when enumerating
primitive recursive functions.

\begin{prob}
  Give the complete primitive recursive notation for~$\Mult$.
\end{prob}

\end{document}
