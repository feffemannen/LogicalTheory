% Part: many-valued-logic
% Chapter: three-valued-logics
% Section: lukasiewicz

\documentclass[../../../include/open-logic-section]{subfiles}

\begin{document}

\olfileid{mvl}{inf}{god}

\olsection{G\"odel logics}

\begin{editorial}
  This is a short ``stub'' of a section on infinite-valued G\"odel logic.
\end{editorial}

\begin{defn}\ollabel{def:goedel} Infinite-valued G\"odel
logic~$\LogGod[\infty]$ is defined using the matrix:
\begin{enumerate}
  \item The standard propositional language $\Lang L_0$ with
  $\lfalse$, $\lnot$, $\land$, $\lor$, $\lif$.
  \item The set of truth values $V_\infty$.
  \item $1$ is the only designated value, i.e., $V^+ = \{1\}$.
  \item Truth functions are given by the following functions:
  \begin{align*}
    \tf{\lfalse} & = 0\\
    \tf{\lnot}[\LogGod](x) & = \begin{cases}
      $1$ & \text{if } x =0\\
      $0$ & \text{otherwise}
    \end{cases}\\
    \tf{\land}[\LogGod](x,y) & = \min(x,y)\\
    \tf{\lor}[\LogGod](x,y) & = \max(x,y)\\
    \tf{\lif}[\LogGod](x,y) & = \begin{cases}
      1 & \text{if } x \le y\\
      y & \text{otherwise.}
    \end{cases}
    \end{align*}
\end{enumerate}
$m$-valued G\"odel logic is defined the same, except $V = V_m$.
\end{defn}

\begin{prop}
  The logic $\LogGod[3]$ defined by \olref[thr][god]{defn:goedel}
  is the same as $\LogGod[3]$ defined by \olref{def:goedel}.
\end{prop}

\begin{proof}
  This can be seen by comparing the truth tables for the connectives
  given in \olref[thr][god]{defn:goedel} with the truth tables
  determined by the equations in \olref{def:goedel}:
  \begin{center}
    \begin{tabular}{c|c} 
      $\tf{\lnot}[\LogGod[3]]$ & \\ 
      \hline  
      $1$ & $0$ \\ 
      $1/2$ & $0$ \\
      $0$ & $1$ 
    \end{tabular}
    \quad
    \begin{tabular}{c|ccc} 
      $\tf{\land}[\LogGod]$ & $1$ & $1/2$ & $0$ \\ 
      \hline 
      $1$ & $1$ & $1/2$ & $0$ \\ 
      $1/2$ & $1/2$ & $1/2$ & $0$\\ 
      $0$ & $0$ & $0$ & $0$ 
    \end{tabular}
    \\[2ex]
    \begin{tabular}{c|ccc} 
      $\tf{\lor}[\LogGod]$ & $1$ & $1/2$ & $0$ \\ 
      \hline 
      $1$ & $1$ & $1$ & $1$ \\ 
      $1/2$ & $1$ & $1/2$ & $1/2$ \\
      $0$ & $1$ & $1/2$ & $0$ 
    \end{tabular}
    \quad
    \begin{tabular}{c|ccc} 
      $\tf{\lif}[\LogGod]$ & $1$ & $1/2$ & $0$ \\ 
      \hline 
      $1$ & $1$ & $1/2$ & $0$ \\ 
      $1/2$ & $1$ & $1$ & $0$  \\ 
      $0$ & $1$ & $1$ & $1$ 
    \end{tabular}
  \end{center} 
\end{proof}

\begin{prop}\ollabel{prop:god-infty-m}
  If $\Gamma \Entails[\LogGod[\infty]] !B$ then $\Gamma
  \Entails[\LogGod[m]] !B$ for all~$m \ge 2$.
\end{prop}

\begin{proof}
  Exercise.
\end{proof}

\begin{prob}
  Prove \olref[mvl][inf][god]{prop:god-infty-m}.
\end{prob}

In fact, the converse holds as well.

Like $\LogGod[3]$, $\LogGod[\infty]$ has all intuitionistically valid
!!{formula}s as tautologies, and the same examples of non-tautologies
are non-tautologies of~$\LogGod[\infty]$:
\begin{align*}
  & p \lor \lnot p && (p \lif q) \lif (\lnot p \lor q) \\
  & \lnot\lnot p \lif p && \lnot(p \land q) \lif (\lnot p \lor \lnot q) \\
  &&& ((p \lif q) \lif p) \lif p
\end{align*}
The example of an intuitionistically invalid !!{formula} that is
nevertheless a tautology of~$\LogGod[3]$, $(p \lif q) \lor (q \lif
p)$, is also a tautology in~$\LogGod[\infty]$. In fact,
$\LogGod[\infty]$ can be characterized as intuitionistic logic to
which the schema $(!A \lif !B) \lor (!B \lif !A)$ is added. This was
shown by Michael Dummett, and so $\LogGod[\infty]$ is often referred to
as G\"odel--Dummett logic~$\Log{LC}$.

\begin{prob}
  Show that $(p \lif q) \lor (q \lif p)$ is a
  tautology of~$\LogGod[\infty]$.
\end{prob}

\end{document}