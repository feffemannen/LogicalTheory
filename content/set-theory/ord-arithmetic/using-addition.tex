\documentclass[../../../include/open-logic-section]{subfiles}

\begin{document}

\olfileid{sth}{ord-arithmetic}{using-addition}
\olsection{Using Ordinal Addition}

Using addition on the ordinals, we can explicitly calculate the ranks
of various sets, in the sense of \olref[spine][rank]{defnsetrank}:

\begin{lem}\ollabel{rankcomputation}
If $\setrank{A} = \alpha$ and $\setrank{B} = \beta$, then:
\begin{enumerate}
	\item\ollabel{exrankpow} $\setrank{\Pow{A}} = \alpha\ordplus 1$
	\item\ollabel{exrankpair} $\setrank{\{A, B\}} = \max(\alpha,
	\beta) \ordplus 1$
	\item\ollabel{exrankcup} $\setrank{A \cup B} = \max(\alpha,
	\beta)$
	\item\ollabel{exranktuple} $\setrank{\tuple{A,B}} = \max(\alpha,
	\beta) \ordplus  2$
	\item\ollabel{exranktimes} $\setrank{A \times B} \leq \max(\alpha,
	\beta) \ordplus  2$
	\item\ollabel{exrankunion} $\setrank{\bigcup A} = \alpha$ when
	$\alpha$ is empty or a limit; $\setrank{\bigcup A} = \gamma$ when
	$\alpha = \gamma\ordplus 1$
\end{enumerate}
\end{lem}

\begin{proof}
Throughout, we invoke \olref[spine][rank]{ranksupstrict}
repeatedly.

\emph{\olref{exrankpow}.} If $x \subseteq A$ then $\setrank{x} \leq
\setrank{A}$. So $\setrank{\Pow{A}} \leq \alpha \ordplus  1$. Since $A
\in \Pow{A}$ in particular, $\setrank{\Pow{A}} = \alpha \ordplus  1$.

\emph{\olref{exrankpair}.} By \olref[spine][rank]{ranksupstrict}

\emph{\olref{exrankcup}.} By \olref[spine][rank]{ranksupstrict}.

\emph{\olref{exranktuple}.} By \olref{exrankpair}, twice.

\emph{\olref{exranktimes}.} Note that $A \times B \subseteq
\Pow{\Pow{A \cup B}}$, and invoke \olref{exranktuple}. 

\emph{\olref{exrankunion}.} If $\alpha = \gamma\ordplus 1$, there is
some $c \in A$ with $\setrank{c} = \gamma$, and no !!{element} of $A$
has higher rank; so $\setrank{\bigcup A} = \gamma$. If $\alpha$ is a
limit ordinal, then $A$ has !!{element}s with rank arbitrarily close
to (but strictly less than) $\alpha$, so that $\bigcup A$ also has
!!{element}s with rank arbitrarily close to (but strictly less than)
$\alpha$, so that $\setrank{\bigcup A} = \alpha$.
\end{proof}
\noindent
We leave it as an exercise to show why \olref{exranktimes} involves an
\emph{in}equality.

\begin{prob}
Produce sets $A$ and $B$ such that $\setrank{A \times B}=
\max(\setrank{A}, \setrank{B})$. Produce sets $A$ and $B$ such that
$\setrank{A \times B}\max(\setrank{A}, \setrank{B}) \ordplus  2$. Are
any other ranks possible?
\end{prob}

We are also now in a position to show that several reasonable notions
of what it might mean to describe an ordinal as ``finite'' or ``infinite'' coincide:

\begin{lem}\ollabel{ordinfinitycharacter}
For any ordinal $\alpha$, the following are equivalent:
\begin{enumerate}
	\item\ollabel{ord:notinomega} $\alpha\notin \omega$, i.e.,
	$\alpha$ is not a natural number
	\item\ollabel{ord:omegaplus} $\omega \leq \alpha$ 
	\item\ollabel{ord:oneplus} $1 \ordplus  \alpha = \alpha$ 
	%\alpha \approx \alpha \ordplus 1$
	\item\ollabel{ord:plusone} $\alpha \approx \alpha\ordplus 1$,
	i.e., $\alpha$ and $\alpha\ordplus 1$ are equinumerous
	\item\ollabel{ord:infinite} $\alpha$ is Dedekind infinite	
	\end{enumerate}
\end{lem}
\noindent
So we have five provably equivalent ways to understand what it takes for an ordinal to be (in)finite.

\begin{proof}
\emph{\olref{ord:notinomega} $\Rightarrow$ \olref{ord:omegaplus}.} By
Trichotomy. 

\emph{\olref{ord:omegaplus} $\Rightarrow$ \olref{ord:oneplus}.} Fix
$\alpha \geq \omega$. By Transfinite Induction, there is some least
ordinal $\gamma$ (possibly $0$) such that there is a limit ordinal
$\beta$ with $\alpha = \beta \ordplus \gamma$. Now:
\[
	1 \ordplus \alpha =  
	1 \ordplus (\beta \ordplus \gamma) = 
	(1 \ordplus \beta) \ordplus \gamma =  
	\supstrict_{\delta < \beta} (1 \ordplus  \delta) \ordplus  \gamma = 
	\beta \ordplus  \gamma = 
	\alpha.
\]
\emph{\olref{ord:oneplus} $\Rightarrow$ \olref{ord:plusone}.} There is
clearly !!a{bijection} $f \colon (\alpha \disjointsum 1) \to (1
\disjointsum \alpha)$. If $1 \ordplus \alpha = \alpha$, there is an
isomorphism $g \colon (1 \disjointsum \alpha) \to \alpha$. Now
consider $\comp{f}{g}$.

\emph{\olref{ord:plusone} $\Rightarrow$ \olref{ord:infinite}.} If
$\alpha \approx \alpha \ordplus 1$, there is !!a{bijection} $f \colon
(\alpha \disjointsum 1) \to \alpha$. Define $g(\gamma) = f(\gamma, 0)$
for each $\gamma < \alpha$; this !!{injection} witnesses that $\alpha$
is Dedekind infinite, since $f(0,1) \in \alpha \setminus \ran{g}$. 

\emph{\olref{ord:infinite} $\Rightarrow$ \olref{ord:notinomega}.} This
is \olref[z][infinity-again]{naturalnumbersarentinfinite}.
\end{proof}

\end{document}