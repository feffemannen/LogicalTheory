\documentclass[../../../include/open-logic-section]{subfiles}

\begin{document}

\olfileid{sth}{ord-arithmetic}{add}
\olsection{Ordinal Addition}

Suppose we want to add $\alpha$ and $\beta$. We can simply put a
{copy} of $\beta$ immediately after a copy of $\alpha$. (We need to
take \emph{copies}, since we know from
\olref[ordinals][basic]{ordinalsaresubsets} that either $\alpha
\subseteq \beta$ or $\beta \subseteq \alpha$.) The intuitive effect of
this is to run through an $\alpha$-sequence of steps, \emph{and then}
to run through a $\beta$-sequence. The resulting sequence will be
well-ordered; so by
\olref[ordinals][ordtype]{thmOrdinalRepresentation} it is isomorphic
to a (unique) ordinal. That ordinal can be regarded as the \emph{sum}
of $\alpha$ and $\beta$. 

That is the intuitive idea behind ordinal addition. To define it
rigorously, we start with the idea of taking \emph{copies} of sets.
The idea here is to use arbitrary tags, $0$ and $1$, to keep track of
which object came from where:

\begin{defn}\ollabel{defdissum}
The \emph{disjoint sum} of $A$ and $B$ is $A \disjointsum B = (A\times
\{0\}) \cup (B \times \{1\})$.
\end{defn}

We next define an ordering on pairs of ordinals:

\begin{defn}
For any ordinals $\alpha_1, \alpha_2, \beta_1, \beta_2$, say that:
\begin{align*}
	\tuple{\alpha_1, \alpha_2} \rlexless \tuple{\beta_1, \beta_2}\text{ iff }& 
	\text{either $\alpha_2 \in \beta_2$}\\
	& \text{or both $\alpha_2 = \beta_2$ and $\alpha_1 \in \beta_1$}
\end{align*} 
\end{defn}

This is a \emph{reverse lexicographic} ordering, since you order by
the second element, then by the first. Now recall that we wanted to
define $\alpha \ordplus \beta$ as the order type of a copy of $\alpha$
followed by a copy of $\beta$. To achieve that, we say:

\begin{defn}\ollabel{defordplus}
For any ordinals $\alpha$, $\beta$, their sum is $\alpha \ordplus
\beta = \ordtype{\alpha \disjointsum \beta, \rlexless}$.
\end{defn}
\noindent
Note that we slgihtly abused notation here; strictly we should write ``$\Setabs{\tuple{x,y}\in \alpha\disjointsum\beta}{x \rlexless y}$'' in place of ``$\rlexless$''. For brevity, though, we will continue to abuse notation in this way in what follows. 

The following result, together with
\olref[ordinals][ordtype]{thmOrdinalRepresentation}, confirms that our
definition is well-formed:

\begin{lem}\ollabel{ordsumlessiswo} 
$\tuple{\alpha \disjointsum \beta, \rlexless}$ is a well-order, for
any ordinals $\alpha$ and $\beta$.
\end{lem}

\begin{proof}
Obviously $\rlexless$ is connected on $\alpha \disjointsum \beta$. To
show it is well-founded, fix a non-empty $X \subseteq \alpha
\disjointsum \beta$. Let $Y$ be the subset of $X$ whose second coordinate is as small as possible, i.e.\ $Y = \Setabs{\tuple{\gamma, i} \in X}{(\forall \tuple{\delta, j} \in X)i \leq j}$. Now choose the element of $Y$ with smallest first coordinate. 
\end{proof}
\noindent 
So we have a nice, explicit definition of ordinal addition. Here is
an unsurprising fact (recall that  $1 = \{0\}$, by
\olref[z][infinity-again]{defnomega}):
\begin{prop}
$\alpha \ordplus  1 = \ordsucc{\alpha}$, for any ordinal $\alpha$.
\end{prop}

\begin{proof}
Consider the isomorphism $f$ from $\ordsucc{\alpha} = \alpha \cup
\{\alpha\}$ to $\alpha\disjointsum1 = (\alpha \times \{0\})
\disjointsum (\{0\} \times \{1\})$ given by $f(\gamma) =
\tuple{\gamma, 0}$ for $\gamma \in \alpha$, and $f(\alpha) = \tuple{0,
1}$.
\end{proof}
\noindent
Moreover, it is easy to show that addition obeys certain recursive
conditions:

\begin{lem}\ollabel{ordadditionrecursion}
For any ordinals $\alpha, \beta$, we have:
\begin{align*}
	\alpha\ordplus 0 &= \alpha\\
	\alpha \ordplus  (\beta\ordplus 1) &= (\alpha \ordplus  \beta) \ordplus  1\\
	\alpha  \ordplus  \beta &= \supstrict_{\delta < \beta}(\alpha \ordplus  \delta) && \text{if $\beta $ is a limit ordinal}
\end{align*}
\end{lem}

\begin{proof}
We check case-by-case; first:
\begin{align*}
	\alpha \ordplus  0 
	& = \ordtype{(\alpha \times \{0\}) \cup (0 \times \{1\}), \rlexless} \\
	&= \ordtype{(\alpha \times \{0\}) \cup \{0\}, \rlexless}\\
	&= \alpha\\
	\alpha \ordplus (\beta \ordplus  1) 
	%&= \ordtype{(\alpha\times \{0\}) \cup (\ordtype{\beta \ordplus  1}\times \{1\}), \rlexless} \\
	&= \ordtype{(\alpha\times \{0\}) \cup (\ordsucc{\beta}\times \{1\}), \rlexless} \\
	&= \ordtype{(\alpha\times \{0\}) \cup (\beta \times \{1\}), \rlexless} \ordplus 1\\
	&= (\alpha \ordplus  \beta) \ordplus  1
\end{align*}
Now let $\beta \neq \emptyset$ be a limit. If $\delta < \beta$ then
also $\delta\ordplus 1 < \beta$, so $\alpha \ordplus  \delta$ is a
proper initial segment of $\alpha \ordplus  \beta$. So $\alpha
\ordplus  \beta$ is a strict upper bound on $X = \Setabs{\alpha
\ordplus  \delta}{\delta < \beta}$. Moreover, if $\alpha \leq \gamma <
\alpha \ordplus  \beta$, then clearly $\gamma = \alpha \ordplus
\delta$ for some $\delta < \beta$. So $\alpha \ordplus  \beta =
\supstrict_{\delta< \beta}(\alpha\ordplus \delta)$.
\end{proof}

But here is a striking fact. To define ordinal addition, we could
\emph{instead} have simply used the Transfinite Recursion Theorem, and
laid down the recursion equations, exactly as given in
\olref{ordadditionrecursion} (though using ``$\ordsucc{\beta}$''
rather than ``$\beta \ordplus 1$'').

There are, then, two different ways to define operations on the
ordinals. We can define them \emph{synthetically}, by explicitly
constructing a well-ordered set and considering its order type. Or we
can define them \emph{recursively}, just by laying down the recursion
equations. Done correctly, though, the outcome is identical. For
\olref[ordinals][ordtype]{thmOrdinalRepresentation} guarantees
that these recursion equations pin down \emph{unique} ordinals.

In many ways, ordinal arithmetic behaves just like addition of the
natural numbers. For example, we can prove the following:

\begin{lem}\ollabel{ordinaladditionisnice}
If $\alpha, \beta, \gamma$ are ordinals, then:
\begin{enumerate}
	\item\ollabel{ordaddition1} if $\beta < \gamma$, then $\alpha
	\ordplus  \beta < \alpha \ordplus  \gamma$
	\item\ollabel{ordaddition2} if $\alpha \ordplus  \beta =
	\alpha\ordplus \gamma$, then $\beta = \gamma$
	\item\ollabel{ordaddition3}  $\alpha \ordplus  (\beta \ordplus
	\gamma) = (\alpha \ordplus  \beta) \ordplus  \gamma$, i.e.,
	addition is associative
	\item\ollabel{ordaddition4}  If $\alpha \leq \beta$, then $\alpha
	\ordplus  \gamma \leq \beta \ordplus \gamma$
\end{enumerate}
\end{lem}

\begin{proof}
We prove \olref{ordaddition3}, leaving the rest as an exercise. The
proof is by Simple Transfinite Induction on $\gamma$, using
\olref{ordadditionrecursion}. When $\gamma = 0$:
\[
(\alpha \ordplus  \beta) \ordplus  0 = \alpha \ordplus  \beta  = \alpha \ordplus  (\beta \ordplus  0)
\]
When $\gamma = \delta\ordplus 1$, suppose for induction that $(\alpha
\ordplus  \beta) \ordplus  \delta = \alpha \ordplus  (\beta \ordplus
\delta)$; now using \olref{ordadditionrecursion} three times:
\begin{align*}
	(\alpha \ordplus  \beta) \ordplus  (\delta \ordplus  1) & = ((\alpha \ordplus  \beta) \ordplus  \delta)\ordplus 1\\
	& = (\alpha \ordplus  (\beta \ordplus  \delta)) \ordplus  1\\
	& = \alpha \ordplus  ((\beta \ordplus  \delta)\ordplus 1)\\
	& = \alpha \ordplus  (\beta \ordplus  (\delta\ordplus 1))
\end{align*}	
When $\gamma$ is a limit ordinal, suppose for induction that if
$\delta \in \gamma$ then $(\alpha \ordplus  \beta) \ordplus  \delta =
\alpha \ordplus  (\beta \ordplus  \delta)$; now:
\begin{align*}
	(\alpha \ordplus  \beta) \ordplus  \gamma & = \supstrict_{\delta < \gamma}((\alpha \ordplus  \beta) \ordplus  \delta) \\
	&= \supstrict_{\delta < \gamma}(\alpha \ordplus  (\beta \ordplus  \delta))\\
	&= \alpha \ordplus  \supstrict_{\delta < \gamma}(\beta \ordplus  \delta)\\
	& = \alpha \ordplus  (\beta \ordplus  \gamma)
\end{align*}
\end{proof}

\begin{prob}
Prove the remainder of
\olref[sth][ord-arithmetic][add]{ordinaladditionisnice}.
\end{prob}

In these ways, ordinal addition should be very familiar. But, there is a crucial way in which ordinal addition is \emph{not} like addition on the natural numbers.

\begin{prop}\ollabel{ordsumnotcommute}
Ordinal addition is {not} commutative; $1 \ordplus  \omega = \omega <
\omega \ordplus  1$.
\end{prop}

\begin{proof}
Note that $1 \ordplus  \omega = \supstrict_{n < \omega} (1 \ordplus n)
= \omega \in \omega \cup \{\omega\} = \ordsucc{\omega} = \omega
\ordplus  1$.
\end{proof}
\noindent 
Whilst this may initially come as a surprise, \emph{it shouldn't}. On
the one hand, when you consider $1 \ordplus  \omega$, you are thinking
about the order type you get by putting an extra element \emph{before}
all the natural numbers. Reasoning as we did with Hilbert's Hotel in
\olref[sfr][infinite][hilbert]{sec}, intuitively, this extra first
element shouldn't make any difference to the overall order type. On
the other hand, when you consider $\omega \ordplus  1$, you are
thinking about the order type you get by putting an extra element
\emph{after} all the natural numbers. And that's a radically different
beast!

\end{document}