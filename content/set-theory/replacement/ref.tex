\documentclass[../../../include/open-logic-section]{subfiles}

\begin{document}

\olfileid{sth}{replacement}{ref}
\olsection{Replacement and Reflection}

Our last attempt to justify Replacement, via \stagesinex, begins with a deep and lovely result:\footnote{A reminder: all formulas can have parameters (unless explicitly stated otherwise).}
%In this, I will use overlining, such as $x_1, \ldots, x_n$, to
%abbreviate ``$x_1, \ldots, x_n$'': 
\begin{thm}[Reflection Schema]\ollabel{reflectionschema}
For any formula $\phi$:
\[
\forall \alpha \exists \beta > \alpha (\forall x_1 \ldots, x_n \in
V_\beta)(\phi(x_1, \ldots, x_n) \liff \phi^{V_\beta}(x_1, \ldots, x_n))
\]
\end{thm}
\noindent 
As in \olref[sth][replacement][strength]{formularelativization}, $\phi^{V_\beta}$ is the result of restricting every
quantifier in $\phi$ to the set~$V_\beta$. So, intuitively, Reflection
says this: if $\phi$ is true in the entire hierarchy, then $\phi$ is
true in arbitrarily many \emph{initial segments} of the hierarchy. 

\citet{Montague1961} and \citet{Levy1960} showed that (suitable
formulations of) Replacement and Reflection are equivalent,
modulo~$\Z$, so that adding either gives you~$\ZF$. (We prove these results in \olref[sth][replacement][refproofs]{sec}.) Given this
equivalence, one might hope to justify Reflection  and Replacement via
\stagesinex{} as follows: given \stagesinex, the hierarchy should be
very, very tall; so tall, in fact, that nothing we can say about it is
sufficient to bound its height. And we can understand this as the
thought that, if any sentence~$\phi$ is true in the entire hierarchy,
then it is true in arbitrarily many initial segments of the hierarchy.
And that is just Reflection. 

Again, this seems like a genuinely promising attempt to provide an
intrinsic justification for Replacement. But there is much too much to
say about it here. You must now decide for yourself whether it
succeeds.\footnote{Though you might like to continue by reading \citet[95--100]{Incurvati2020}.}

\end{document}