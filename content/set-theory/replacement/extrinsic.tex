\documentclass[../../../include/open-logic-section]{subfiles}

\begin{document}

\olfileid{sth}{replacement}{extrinsic}
\olsection[Extrinsic Considerations]{Extrinsic Considerations about Replacement}

We start by considering an \emph{extrinsic} attempt to justify
Replacement. Boolos suggests one, as follows. 
\begin{quote}
  [\ldots] the reason for adopting the axioms of replacement is quite
  simple: they have many desirable consequences and (apparently) no
  undesirable ones. In addition to theorems about the iterative
  conception, the consequences include a satisfactory if not ideal
  theory of infinite numbers, and a highly desirable result that
  justifies inductive definitions on well-founded relations.
  \citep[229]{Boolos1971}
\end{quote}		
The gist of Boolos's idea is that we should justify Replacement by its
fruits. And the specific fruits he mentions are the things we have
discussed in the past few chapters. Replacement allowed us to prove
that the von Neumann ordinals were excellent surrogates for the idea
of a well-ordering type (this is our ``satisfactory if not ideal
theory of infinite numbers''). Replacement also allowed us to define
the $V_\alpha$s, establish the notion of rank, and prove
$\in$-Induction (this amounts to our ``theorems about the iterative
conception''). Finally, Replacement allows us to prove the Transfinite
Recursion Theorem (this is the ``inductive definitions on well-founded
relations''). 

These are, indeed, desirable consequences. But do these desirable
consequences suffice to \emph{justify} Replacement? \emph{No}. Or at
least, not straightforwardly. 

Here is a simple problem. Whilst we have stated some desirable
consequences of Replacement, we could have obtained many of them via
other means. This is not as well known as it ought to be, though, so we should pause to explain the situation. 

There is a simple theory of sets, Level Theory, or $\LT$ for short.\footnote{The first versions of $\LT$ are offered by \citet{Montague1965} and \citet{Scott1974}; this was simplified, and given a book-length treatment, by \citet{Potter2004}; and \citet{ButtonLT1} has recently simplified $\LT$ further.} $\LT$'s axioms are just Extensionality, Separation, and the claim that every set is a subset of some \emph{level}, where ``level'' is cunningly defined so that the levels behave like our friends, the $V_\alpha$s. So $\ZF$ proves $\LT$; but $\LT$ is \emph{much} weaker than $\ZF$. In fact, $\LT$ does not give you Pairs, Powersets, Infinity, or Replacement. Let $\Zr$ be the result of adding Infinity and Powersets to $\LT$; this delivers Pairs too, so, $\Zr$ is at least as strong as $\Z$. But, in fact, $\Zr$ is strictly stronger than $\Z$, since it adds the claim that every set has a rank (hence my suggestion that we call it $\Zr$). Indeed, $\Zr$ delivers: a perfectly satisfactory theory of ordinals;
results which stratify the hierarchy into well-ordered stages; a proof
of $\in$-Induction; and a \emph{version} of Transfinite Recursion. 

In
short: although Boolos didn't know this, all of the desirable
consequences which he mentions could have been arrived at
\emph{without} Replacement; he simply needed to use $\Zr$ rather than $\Z$. 

(Given all of this, why did we follow the conventional route, of
teaching you $\ZF$, rather than $\LT$ and $\Zr$? There are two reasons. First: for purely historical reasons, starting with $\LT$ is rather nonstandard; we wanted to
equip you to be able to read more standard discussions of set theory. Second: when you are ready to
appreciate $\LT$ and $\Zr$, you can simply read \citealt{Potter2004} and \citealt{ButtonLT1}.)

Of course, since $\Zr$ is strictly weaker than $\ZF$, there are results which
$\ZF$ proves which $\Zr$ leaves open. So one could try to justify
Replacement on extrinsic grounds by pointing to one of these results.
But, once you know how to use $\Zr$, it is quite hard to find many
examples of things that are (a) settled by Replacement but not
otherwise, and (b) are intuitively true. (For more on this, see
\citealt[\S13.2]{Potter2004}.)

The bottom line is this. To provide a compelling extrinsic
justification for Replacement, one would need to find a result which
\emph{cannot} be achieved without Replacement. And that's not an easy
enterprise. 

Let's consider a further problem which arises for any attempt to offer
a purely extrinsic justification for Replacement. (This problem is
perhaps more fundamental than the first.) Boolos does not just point
out that Replacement has many desirable consequences. He also states
that Replacement has ``(apparently) no undesirable'' consequences. But
this paranthetical caveat, ``apparently,'' is surely absolutely
crucial.

Recall how we ended up here: Na\"ive Comprehension ran into
inconsistency, and we responded to this inconsistency by embracing the
cumulative-iterative conception of set. This conception comes equipped
with a story which, we hope, assures us of its consistency. But if we
cannot justify Replacement from within that story, then we have (as
yet) no reason to believe that $\ZF$ is consistent. Or rather: we have
no reason to believe that $\ZF$ is consistent, apart from the (perhaps
merely contingent) fact that no one has discovered a contradiction
\emph{yet}. In exactly that sense, Boolos's comment seems to come down
to this: ``(apparently) $\ZF$ is consistent''. We should demand
greater reassurance of consistency than this. 

This issue will affect any \emph{purely} extrinsic attempt to justify
Replacement, i.e., any justification which is couched solely in terms
of the (known) consequences of $\ZF$. As such, we will want to look
for an \emph{intrinsic} justification of Replacement, i.e., a
justification which suggests that the story which we told about sets
somehow ``already'' commits us to Replacement. 

\end{document}