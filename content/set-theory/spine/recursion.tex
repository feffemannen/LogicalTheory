\documentclass[../../../include/open-logic-section]{subfiles}

\begin{document}

\olfileid{sth}{spine}{recursion}
\olsection{The Transfinite Recursion Theorem(s)}

The first thing we must do, though, is confirm that
\olref[valpha]{defValphas} is a successful definition. More generally,
we need to prove that any attempt to offer a transfinite by
(transfinite) recursion will succeed. That is the aim of this section.

\emph{Warning: this is tricky material}. The overarching moral,
though, is quite simple: Transfinite Induction plus Replacement
guarantee the legitimacy of (several versions of) transfinite
recursion.\footnote{A reminder: all formulas and terms can have parameters (unless explicitly stated otherwise).}

\begin{defn}
	Let $\tau(x)$ be a term; let $f$ be a function; let $\gamma$ be an ordinal. We say that $f$ is an \emph{$\alpha$-approximation} for $\tau$ iff both $\dom{f} = \alpha$ and $(\forall \beta \in \alpha)f(\beta) = \tau(\funrestrictionto{f}{\beta})$.
\end{defn}

\begin{lem}[Bounded Recursion]\ollabel{transrecursionfun}
For any term $\tau(x)$ and any ordinal $\alpha$, there is a unique $\alpha$-approximation for $\tau$.
\end{lem}
\begin{proof}
We will show that, for any $\gamma \leq \alpha$, there is a unique
$\gamma$-approximation. 

We first establish uniqueness. Let $g$ and $h$ (respectively) be $\gamma$- and $\delta$-approximations. A transfinite induction on their arguments shows that
$g(\beta) = h(\beta)$ for any $\beta \in \dom{g} \cap \dom{h} =
\gamma \cap \delta \min(\gamma, \delta)$. So our approximations are unique (if they exist), and agree on all values.

To establish existence, we now use a simple transfinite induction
(\olref[ordinals][opps]{simpletransrecursion}) on ordinals
$\delta \leq \alpha$. 

The empty function is trivially an $\emptyset$-approximation. 

If $g$ is a $\gamma$-approximation, then $g \cup
\{\tuple{\ordsucc{\gamma}, \tau(g)}\}$ is a $\ordsucc{\gamma}$-approximation.

If $\gamma$ is a limit ordinal and $g_\delta$ is a $\delta$-approximation for all $\delta < \gamma$, let $g = \bigcup_{\delta \in \gamma} g_\delta$. This
is a function, since our various $g_\delta$s agree on all values. And
if $\delta \in \gamma$ then $g(\delta) = g_{\ordsucc{\delta}}(\delta) =
\tau(\funrestrictionto{g_{\ordsucc{\delta}}}{\delta}) =
\tau(\funrestrictionto{g}{\delta})$.

This completes the proof by transfinite induction.
\end{proof}

If we allow ourselves to define a \emph{term} rather than a function,
then we can remove the bound $\alpha$ from the previous result. In
the statement and proof of the following result, when $\sigma$ is a term, we
let $\funrestrictionto{\sigma}{\alpha} = \Setabs{\tuple{\beta,
\sigma(\beta)}}{\beta \in \alpha}$.

\begin{thm}[General Recursion]\ollabel{transrecursionschema}
For any term $\tau(x)$, we can explicitly define a term
$\sigma(x)$, such that
$\sigma(\alpha) = \tau(\funrestrictionto{\sigma}{\alpha})$ for any
ordinal $\alpha$. 	
\end{thm}

\begin{proof}
For each $\alpha$, by \olref{transrecursionfun} there is a unique $\alpha$-approximation, $f_\alpha$, for $\tau$. Define $\sigma(\alpha)$ as $f_{\ordsucc{\alpha}}(\alpha)$. Now:
	\begin{align*}
		\sigma(\alpha) &= 
		f_{\ordsucc{\alpha}}(\alpha) \\&= 
		\tau(\funrestrictionto{f_{\ordsucc{\alpha}}}{\alpha}) \\&= 
		\tau(\Setabs{\tuple{\beta, f_{\ordsucc{\alpha}}(\beta)}}{\beta \in \alpha}) \\&= 
		\tau(\Setabs{\tuple{\beta, f_{\alpha}(\beta)}}{\beta \in \alpha})\\&=
		\tau(\funrestrictionto{\sigma}{\alpha})
	\end{align*}
noting that $f_{\alpha}(\beta) = f_{\ordsucc{\alpha}}(\beta)$ for all $\beta < \alpha$, as in \olref{transrecursionfun}. 
\end{proof}
\noindent 
Note that \olref{transrecursionschema} is a \emph{schema}. Crucially, we cannot
expect $\sigma$ to define a function, i.e., a certain kind of
\emph{set}, since then $\dom{\sigma}$ would be the set of all
ordinals, contradicting the Burali-Forti Paradox
(\olref[ordinals][basic]{buraliforti}).

It still remains to show, though, that \olref{transrecursionschema}
vindicates our definition of the $V_\alpha$s. This may not be
immediately obvious; but it will become apparent with a last, simple,
version of transfinite recursion.

\begin{thm}[Simple Recursion]\ollabel{simplerecursionschema} 
For any terms $\tau(x)$ and $\theta(x)$ and any set $A$, we can
explicitly define a term $\sigma(x)$ such that:
\begin{align*}
	\sigma(\emptyset) &= A\\
	\sigma(\ordsucc{\alpha}) &= \tau(\sigma(\alpha)) &&
		\text{for any ordinal }\alpha\\
	\sigma(\alpha) &= \theta(\ran{\funrestrictionto{\sigma}{\alpha}})&&
	\text{when }\alpha\text{ is a limit ordinal}
\end{align*}
\end{thm}

\begin{proof}
We start by defining a term, $\xi(x)$, as follows: 
%t $\xi(x) = A$ if $x$ is not a function whose domain is an ordinal;
%otherwise let:
\[
	\xi(x) = 
	\begin{cases}
			A &\text{if $x$ is not a function whose}\\
			&\hspace{1em}\text{domain is an ordinal; otherwise:}\\
			\tau(x(\alpha)) & \text{if $\dom{x} = \ordsucc{\alpha}$}\\
			\theta(\ran{x}) & \text{if $\dom{x}$ is a limit ordinal}
	\end{cases}
\]
By \olref{transrecursionschema}, there is a term $\sigma(x)$ such that
$\sigma(\alpha) = \xi(\funrestrictionto{\sigma}{\alpha})$ for every
ordinal $\alpha$; moreover, $\funrestrictionto{\sigma}{\alpha}$ is a
function with domain $\alpha$. We show that $\sigma$ has the required
properties, by simple transfinite induction
(\olref[ordinals][opps]{simpletransrecursion}). 

First, $\sigma(\emptyset) = \xi(\emptyset) = A$. 

Next, $\sigma(\ordsucc{\alpha}) = \xi(\funrestrictionto{\sigma}{\ordsucc{\alpha}}) = \tau(\funrestrictionto{\sigma}{\ordsucc{\alpha}}(\alpha)) = \tau(\sigma(\alpha))$.

Last, $\sigma(\alpha) = \xi(\funrestrictionto{\sigma}{\alpha}) = \theta(\ran{\funrestrictionto{\sigma}{\alpha}})$, when $\alpha$ is a limit.
\end{proof}
\noindent
Now, to vindicate \olref[valpha]{defValphas}, just take $A
= \emptyset$ and $\tau(x) = \Pow{x}$ and $\theta(x) = \bigcup x$. At long last, this vindicates the definition of the $V_\alpha$s!{}

\end{document}