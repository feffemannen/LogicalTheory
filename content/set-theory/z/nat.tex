\documentclass[../../../include/open-logic-section]{subfiles}

\begin{document}
	
\olfileid{sth}{z}{nat}
\olsection{Selecting our Natural Numbers}

%Let us take it for granted that the informal principle \stagesinf{}
%can be justified. It is also, though, worth commenting on the
%particular axiom we extracted from that informal principle. 

In \olref[infinity-again]{defnomega}, we explicitly defined
the expression ``natural numbers''. How should you understand this
stipulation? It is not a metaphysical claim, but just a decision to
\emph{treat} certain sets as the natural numbers. We touched upon reasons for thinking this in
\olref[sfr][rel][ref]{sec}, \olref[sfr][arith][ref]{sec} and
\olref[sfr][infinite][dedekindsproof]{sec}. But we can make these
reasons even more pointed.

Our Axiom of Infinity follows \citet{VonNeumann1925}. But here is
another axiom, which we could have adopted instead:

\begin{defish}
\emph{Zermelo's \citeyear{Zermelo1908Untersuchungen} Axiom of
Infinity.} There is a set $A$ such that $\emptyset \in A$ and
$(\forall x \in A)\{x\} \in A$. 
\end{defish}

Had we used Zermelo's axiom, instead of our (von Neumann-inspired)
Axiom of Infinity, we would equally well have been given a Dedekind
infinite set, and so a Dedekind algebra. On Zermelo's approach, the
distinguished element of our algebra would again have been $\emptyset$
(our surrogate for $0$), but the injection would have been given by
the map $x \mapsto \{x\}$, rather than $x \mapsto x \cup \{x\}$. The
simplest upshot of this is that Zermelo treats $2$ as
$\{\{\emptyset\}\}$, whereas we (with von Neumann) treat $2$ as
$\{\emptyset, \{\emptyset\}\}$. 

Why choose one axiom of Infinity rather than the other? The main
practical reason is that von Neumann's approach ``scales up'' to
handle transfinite numbers rather well. We will explore this from
\olref[ordinals][]{chap} onwards. However, from the simple
perspective of \emph{doing arithmetic}, both approaches would do
equally well. So if someone tells you that the natural numbers
\emph{are} sets, the obvious question is: \emph{Which sets are they?} 

This precise question was made famous by \citet{Benacerraf1965}. But
it is worth emphasising that it is just the most famous example of a
phenomenon that we have encountered many times already. The basic
point is this. Set theory gives us a way to \emph{simulate} a bunch of
``intuitive'' kinds of entities: the reals, rationals, integers, and
naturals, yes; but also ordered pairs, functions, and relations.
However, set theory never provides us with a \emph{unique} choice of
simulation. There are \emph{always} alternatives
which---straightforwardly---would have served us just as well. 

\end{document}