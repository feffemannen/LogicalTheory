\documentclass[../../../include/open-logic-section]{subfiles}

\begin{document}

\olfileid{sth}{z}{arbintersections}
\olsection{Closure, Comprehension, and Intersection}

In \olref[sth][z][milestone]{sec}, we suggested that you should look
back through the na\"ive work of \olref[sfr][][]{part} and check that
it can be carried out in~$\Zminus$. If you followed that advice,
\emph{one} point might have tripped you up: the use of
\emph{intersection} in Dedekind's treatment of \emph{closures}. 

Recall from \olref[sfr][infinite][dedekind]{Closure} that
\begin{align*}
  \closureofunder{f}{o} & = \bigcap\Setabs{X}{o \in X \text{ and $X$ is
$f$-closed}}.
\intertext{The general shape of this is a definition of the form:}
  C & = \bigcap\Setabs{X}{\phi(X)}.
\end{align*}
But this should ring alarm bells: since Na\"ive Comprehension fails,
there is no guarantee that $\Setabs{X}{\phi(X)}$ exists. It looks
dangerously, then, like such definitions are \emph{cheating}. 

Fortunately, they are not cheating; or rather, if they \emph{are}
cheating as they stand, then we can engage in some honest toil to
render them kosher. That honest toil was foreshadowed in
\olref[sth][z][sep]{prop:intersectionsexist}, when we explained why
$\bigcap A$ exists for any $A \neq \emptyset$. But we will spell it out
explicitly.

Given Extensionality, if we attempt to define $C$ as
$\bigcap\Setabs{X}{\phi(X)}$, all we are really asking is for an
object $C$ which obeys the following:
\begin{equation}\ollabel{bicondelimarbintersection}
	\forall x(x \in C \liff \forall X(\phi(X) \lif x \in X))
\end{equation}
Now, suppose there is \emph{some} set, $S$, such that $\phi(S)$. Then
to deliver \olref{bicondelimarbintersection}, we can simply define $C$
using \emph{Separation}, as follows:
\[
	C = \Setabs{x \in S}{\forall X(\phi(X) \lif x \in X)}.
\]
We leave it as an exercise to check that this definition yields
\olref{bicondelimarbintersection}, as desired. 
%  fix $x$. First suppose that $x \in C$, as (re)defined; then both $x
%  \in S$ and $\forall X(\phi(X) \lif x \in X)$; but since
%  $\phi(S)$, this reduces to the condition that $\forall X(\phi(X)
%  \lif x \in X)$. Conversely, suppose $\forall X(\phi(X) \lif
%  x \in X)$; then, since $\phi(S)$, we also have that $x \in S$; so
%  $x \in c$, as (re)defined. 
And this general strategy will allow us to circumvent any apparent use
of na\"ive comprehension in defining intersections. In the particular
case which got us started on this, namely that of
$\closureofunder{f}{o}$, here is how that would work. We began the
proof of \olref[sfr][infinite][dedekind]{closureproperties} by noting
that $o \in \ran{f}\cup\{o\}$ and that $\ran{f} \cup \{o\}$ is
$f$-closed. So, we can define what we want thus:
\[
	\closureofunder{f}{o} = \Setabs{x \in \ran{f} \cup \{o\}}{(\forall X \ni o)(X \text{ is $f$-closed} \lif x \in X)}.
\]

\end{document}