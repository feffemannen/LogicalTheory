\documentclass[../../../include/open-logic-section]{subfiles}

\begin{document}

\olfileid{sth}{choice}{banach}
\olsection{The Banach--Tarski Paradox}

We might also attempt to justify Choice, as Boolos attempted to
justify Replacement, by appealing to \emph{extrinsic} considerations
(see \olref[replacement][extrinsic]{sec}). After all, adopting Choice
has many desirable consequences: the ability to compare every
cardinal; the ability to well-order every set; the ability to treat
cardinals as a particular kind of ordinal; etc. 

Sometimes, however, it is claimed that Choice has \emph{undesirable}
consequences. Mostly, this is due to a result by
\cite{BanachTarski1924}. 

\begin{thm}[Banach--Tarski Paradox (in $\ZFC$)]
Any ball can be decomposed into finitely many pieces, which can be
reassembled (by rotation and transportation) to form two copies of
that ball.
\end{thm}
\noindent 
At first glance, this is a bit amazing. Clearly the two balls have
\emph{twice} the volume of the original ball. But rigid
motions---rotation and transportation---do not change volume. So it
looks as if Banach--Tarski allows us to magick new matter into
existence.

It gets worse.\footnote{See \citet[Theorem 3.12]{Wagon2016}.} Similar
reasoning shows that a pea can be cut into finitely many pieces, which
can then be reassembled (by rotation and transportation) to form an
entity the shape and size of Big Ben.

None of this, however, holds in $\ZF$ on its own.\footnote{Though
Banach--Tarski can be proved with principles which are strictly weaker
than Choice; see \citet[303]{Wagon2016}.} So we face a decision:
reject Choice, or learn to live with the ``paradox''. 

We're going to suggest that we should learn to live with the
``paradox''. Indeed, we don't think it's much of a paradox at all. In
particular, we don't see why it is any more or less paradoxical than
any of the following results:\footnote{\citet[276--7]{Potter2004},
\citet[16]{Weston2003}, \citet[31, 308--9]{Wagon2016}, make
similar points, using other examples.}
\begin{enumerate}
	\item There are as many points in the interval $(0,1)$ as in  $\Real$. 
	\\\emph{Proof}: consider $\tan(\pi(r-\nicefrac{1}{2})))$.
	\item There are as many points in a line as in a square.
	\\See \olref[his][set][pathology]{sec} and \olref[his][set][cantorplane]{sec}.
	\item There are space-filling curves. 
	\\See \olref[his][set][pathology]{sec} and \olref[his][set][hilbertcurve]{sec}.
\end{enumerate}
None of these three results require Choice. Indeed, we now just regard
them as surprising, lovely, bits of mathematics. Maybe we should adopt
the same attitude to the Banach--Tarski Paradox.

To be sure, a technical observation is required here; but it only
requires keeping a level head. Rigid motions preserve volume.
Consequently, the five\footnote{We stated the Paradox in terms of
``finitely many pieces''. In fact, \citet{Robinson1947} proved that
the decomposition can be achieved with \emph{five} pieces
(but no fewer). For a proof, see \citet[pp.~66--7]{Wagon2016}.} pieces
into which the ball is decomposed cannot all be \emph{measurable}.
Roughly put, then, it makes no sense to assign a volume to these
individual pieces. You should think of these as unpicturable,
``infinite scatterings'' of points. Now, maybe it is ``weird'' to
conceive of such ``infinitely scattered'' sets. But their existence
seems to fall out from the injunction, embodied in \stagesacc{}, that
you should form \emph{all possible} collections of earlier-available
sets. 

If none of that convinces, here is a final (extrinsic) argument in
favour of embracing the Banach--Tarski Paradox. It immediately entails
the best math joke of all time:
\begin{enumerate}
	\item[] \emph{Question}. What's an anagram of ``Banach--Tarski''? 
	\item[] \emph{Answer}. ``Banach--Tarski Banach--Tarski''.
\end{enumerate}

\end{document}