\documentclass[../../../include/open-logic-section]{subfiles}

\begin{document}

\olfileid{sth}{ordinals}{vn}	
\olsection[Von Neumann's Construction]{Von Neumann's Construction of the Ordinals}

\olref[sth][ordinals][iso]{thm:woalwayscomparable} gives rise to a
thought. We could introduce certain objects, called \emph{order
types}, to go proxy for the well-orderings. Writing $\ordtype{A, <}$
for the order type of the well-ordering $\tuple{A, <}$, we would hope
to secure the following two principles:
\begin{align*}
	\ordtype{A, <} = \ordtype{B, \lessdot} & 
	\text{ iff } \ordeq{\tuple{A, <}}{\tuple{B, \lessdot}}\\
	\ordtype{A, <} < \ordtype{B, \lessdot}&
	\text{ iff }\ordeq{\tuple{A, <}}{\tuple{B_b, \lessdot_b}}\text{ for some }b \in B
\end{align*}
Moreover, we might hope to introduce order-types \emph{as certain
sets}, just as we can introduce the natural numbers as certain sets. 

The most common way to do this---and the approach we will follow---is
to define these order-types via certain \emph{canonical} well-ordered
sets. These canonical sets were first introduced by von Neumann:

\begin{defn}
The set $A$ is \emph{transitive} {iff} $(\forall x \in A)x \subseteq
A$. Then $A$ is an \emph{ordinal} {iff} $A$ is transitive and
well-ordered by $\in$.
\end{defn}

In what follows, we will use Greek letters for ordinals. It follows
immediately from the definition that, if $\alpha$ is an ordinal, then
$\tuple{\alpha, \in_\alpha}$ is a well-ordering, where $\in_\alpha =
\Setabs{\tuple{x, y} \in \alpha^2}{x \in y}$. So, abusing notation a
little, we can just say that $\alpha$ \emph{itself} is a
well-ordering. 

Here are our first few ordinals:
\[
	\emptyset, \{\emptyset\}, 
	\{\emptyset, \{\emptyset\}\}, 
	\{\emptyset, \{\emptyset\}, \{\emptyset, \{\emptyset\}\}\}, \ldots
\]
You will note that these are the first few ordinals that we
encountered in our Axiom of Infinity, i.e., in von Neumann's
definition of $\omega$ (see \olref[sth][z][infinity-again]{sec}). This
is no coincidence. Von Neumann's definition of the ordinals treats
natural numbers as ordinals, but allows for transfinite ordinals too. 

As always, we can now ask: \emph{are} these the ordinals? Or has von
Neumann simply given us some sets that we can \emph{treat} as the
ordinals? The kinds of discussions one might have about this question
are similar to the discussions we had in \olref[sfr][rel][ref]{sec},
\olref[sfr][arith][ref]{sec},
\olref[sfr][infinite][dedekindsproof]{sec}, and
\olref[sth][z][nat]{sec}, so we will not belabour the point.
Instead, in what follows, we will simply use ``the ordinals'' to speak
of ``the von Neumann ordinals''. 

\end{document}