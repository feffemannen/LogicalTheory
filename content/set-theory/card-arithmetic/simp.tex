\documentclass[../../../include/open-logic-section]{subfiles}

\begin{document}
\olfileid{sth}{card-arithmetic}{simp}

\olsection{Simplifying Addition and Multiplication}

It turns out that transfinite cardinal addition and multiplication is
\emph{extremely} easy. This follows from the fact that cardinals are
(certain) ordinals, and so well-ordered, and so can be manipulated in
a certain way. Showing this, though, is \emph{not} so easy. To start,
we need a tricksy definition:

\begin{defn}
We define a \emph{canonical ordering}, $\canonord$, on pairs of
ordinals, by stipulating that $\tuple{\alpha_1, \alpha_2} \canonord
\tuple{\beta_1, \beta_2}$ iff either:
\begin{enumerate}
	\item $\max(\alpha_1, \alpha_2) < \max(\beta_1, \beta_2)$; or
	\item $\max(\alpha_1, \alpha_2) = \max(\beta_1, \beta_2)$ and
	$\alpha_1 < \beta_1$; or
	\item $\max(\alpha_1, \alpha_2) = \max(\beta_1, \beta_2)$ and
	$\alpha_1 = \beta_1$ and $\alpha_2 < \beta_2$
\end{enumerate}
\end{defn}

\begin{lem}
$\tuple{\alpha \times \alpha, \canonord}$ is a well-order, for any
ordinal $\alpha$.\footnote{Cf.\ the naughtiness described in the
footnote to \olref[ord-arithmetic][add]{defordplus}.}
\end{lem}

\begin{proof}
Evidently $\canonord$ is connected on $\alpha \times \alpha$. For
suppose that neither $\tuple{\alpha_1, \alpha_2}$ nor $\tuple{\beta_1,
\beta_2}$ is $\canonord$-less than the other. Then $\max(\alpha_1,
\alpha_2) = \max(\beta_1, \beta_2)$ and $\alpha_1 = \beta_1$ and
$\alpha_2 = \beta_2$, so that $\tuple{\alpha_1, \alpha_2} =
\tuple{\beta_1,  \beta_2}$.

To show well-ordering, let $X \subseteq \alpha\times\alpha$ be
non-empty. Since $\alpha$ is an ordinal, some $\delta$ is the least
member of $\Setabs{\max(\gamma_1, \gamma_2)}{\tuple{\gamma_1,
\gamma_2} \in X}$. Now discard all pairs from
$\Setabs{\tuple{\gamma_1,\gamma_2} \in X}{\max(\gamma_1, \gamma_2) =
\delta}$ except those with least first coordinate; from among these,
the pair with least second coordinate is the $\canonord$-least element
of $X$.
\end{proof}

Now for a teensy, simple observation:

\begin{prop}\ollabel{simplecardproduct}
If $\cardeq{\alpha}{\beta}$, then $\cardeq{\alpha \times \alpha}{\beta
\times \beta}$. 
\end{prop}

\begin{proof}
Just let $f \colon \alpha \to \beta$ induce $\tuple{\gamma_1,
\gamma_2} \mapsto \tuple{f(\gamma_1), f(\gamma_2)}$.
\end{proof}

And now we will put all this to work, in proving a crucial lemma:
\begin{lem}\ollabel{alphatimesalpha}
$\cardeq{\alpha}{\alpha \times \alpha}$, for any infinite ordinal
$\alpha$
\end{lem}

\begin{proof}
For reductio, let $\alpha$ be the least infinite ordinal for which
this is false. \olref[sfr][siz][zigzag]{natsquaredenumerable} shows
that $\cardeq{\omega}{\omega\times\omega}$, so $\omega \in \alpha$.
Moreover, $\alpha$ is a cardinal: suppose otherwise, for reductio;
then $\card{\alpha} \in \alpha$, so that
$\cardeq{\card{\alpha}}{\card{\alpha} \times \card{\alpha}}$, by
hypothesis; and $\cardeq{\card{\alpha}}{\alpha}$ by definition; so
that $\cardeq{\alpha}{\alpha\times\alpha}$ by
\olref{simplecardproduct}. 

Now, for each $\tuple{\gamma_1, \gamma_2} \in \alpha \times \alpha$,
consider the segment:
\begin{align*}
	\text{Seg}(\gamma_1, \gamma_2) &= \Setabs{\tuple{\delta_1, \delta_2} \in \alpha \times \alpha}{\tuple{\delta_1, \delta_2} \canonord \tuple{\gamma_1, \gamma_2}}
\end{align*}
Let $\gamma = \max(\gamma_1, \gamma_2)$. When $\gamma$ is infinite, observe:
\begin{align*}
	\text{Seg}(\gamma_1, \gamma_2) & 
	\precsim ((\gamma \ordplus 1)\ordtimes (\gamma \ordplus 1))
	\text{, by the first clause defining $\canonord$}\\
	&\approx (\gamma \ordtimes \gamma)
	\text{, by \olref[ord-arithmetic][using-addition]{ordinfinitycharacter} and 
	\olref{simplecardproduct}}\\
	&\approx \gamma \text{, by the induction hypothesis}\\
	& \prec \alpha\text{, since $\alpha$ is a cardinal}
\end{align*}
So $\ordtype{\alpha\times \alpha, \canonord} \leq \alpha$, and hence
$\cardle{\alpha \times \alpha}{\alpha}$. Since of course
$\cardle{\alpha}{\alpha \times \alpha}$, the result follows by
Schr\"oder-Bernstein. 
\end{proof}

Finally, we get to our simplifying result:

\begin{thm}\ollabel{cardplustimesmax}
If $\cardfont{a}, \cardfont{b}$ are infinite cardinals, $\cardfont{a}
\cardtimes \cardfont{b} = \cardfont{a} \cardplus \cardfont{b} =
\text{max}(\cardfont{a}, \cardfont{b})$.
\end{thm}

\begin{proof}
Without loss of generality, suppose $\cardfont{a} = \max(\cardfont{a},
\cardfont{b})$. Then invoking \olref{alphatimesalpha},
$\cardfont{a}\cardtimes\cardfont{a} = \cardfont{a} \leq \cardfont{a}
\cardplus \cardfont{b} \leq \cardfont{a} \cardplus \cardfont{a} \leq
\cardfont{a} \cardtimes \cardfont{a}$. \end{proof}\noindent Similarly,
if $\cardfont{a}$ is infinite, an $\cardfont{a}$-sized union of
$\leq\cardfont{a}$-sized sets has size $\leq\cardfont{a}$:

\begin{prop}\ollabel{kappaunionkappasize}
Let $\cardfont{a}$ be an infinite cardinal. For each ordinal $\beta
\in \cardfont{a}$, let $X_\beta$ be a set with $\card{X_\beta} \leq
\cardfont{a}$. Then $\card{\bigcup_{\beta \in \cardfont{a}} X_\beta}
\leq \cardfont{a}$.
\end{prop}

\begin{proof}
For each $\beta \in \cardfont{a}$, fix !!a{injection} $f_\beta \colon
X_\beta \to \cardfont{a}$. Define !!a{injection} $g \colon
\bigcup_{\beta \in \cardfont{a}} X_\beta \to \cardfont{a} \times
\cardfont{a}$ by $g(v) = \tuple{\beta, f_\beta(v)}$, where $v \in
X_\beta$ and $v \notin X_\gamma$ for any $\gamma \in \beta$. Now
$\bigcup_{\beta \in \cardfont{a}} X_\beta \preceq \cardfont{a} \times
\cardfont{a} \approx \cardfont{a}$ by \olref{cardplustimesmax}.
\end{proof}

\end{document}